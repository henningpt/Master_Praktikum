\section{Auswertung}
\label{sec:Auswertung}

Um mit Hilfe der Filmstreifenmethode die Struktur der Probe zu bestimmen, werden
die Radien der "Debye-Ringe", die sich durch die Bestrahlung auf dem Filmstreifen bilden,
vermessen. Durch die Proportionalität zwischen den Radien $r$ und den Beugungswinkeln $\theta$
\begin{align}
\theta=\frac{\SI{360}{\degree}\cdot r}{4 \pi R } \label{eqn:winkel}
\end{align}
mit dem Kameraradius $R$ wird der Beugungswinkel $\theta$ berechnet.
Die Tabelle \ref{tab:bcc} enthält die gemessenen Radien $r$ und die nach \ref{eqn:winkel} berechneten Winkel
für die Metallprobe und die Tablle \ref{tab:salz} die für die Salzprobe.
Es wird davon ausgegangen, dass die "Debye-Ringe" immer zu der ersten Ordnung gehören
und das nur kubische Gitterstrukturen beobachtet werden.
Somit lässt sich die Gleichung \ref{eqn:10} für auftetenden
Reflexe durch \eqref{eqn:5} zu
\begin{align}
\symup{a} = \frac{\lambda \sqrt{h^2 +k^2+l^2}}{2\sin{\theta}} \label{eqn:Gitterkonst}
\end{align}
umformen.
Um aus den berechneten Beugungswinkeln $\theta$ die Gitterkonstante a
zu bestimmen, muss somit auch die zum Refelex gehörige Millerindicesumme
\begin{align*}
  m_{sum}=h^{2} + k^{2} + l^{2}
\end{align*}
der Netzebeneschar (hkl) bekannt sein.
Die Information für welchen Werten von hkl Refexe auftreten
enthält die Strukturamplitude S nach Gleichung \eqref{eqn:Streu}.
Die Gleichung \eqref{eqn:Streu} wiederum benötigt die Basisvektor
der Elementarzelle. Da die der Proben nicht bekannt sind werden
Material typische Gitterstruktur verwendent und die Ergebnisse
untereinander verglichen. Metalle können typischerweise in
 bcc-, fcc- und Diamant-Struktur
vorliegen und Salz in Flourit-, Zinkblende-,
Steinsalz-, Cäsiumchlorid-Struktur.
Durch die entsprechenden Basisvektoren
werden Werte für hkl bestimmt wo die Strukturamplitude
$S\neq0$ und somit ein Refelex auftritt.
Durch die Zweiatomigebasis bei den
Salzstrukturen ist es nötig zwischen zwei Fälle zu differenzieren,
da der Atomformfaktor in Gleichung
\eqref{ean:Streu} nicht ausgeklammert werden kann.
\begin{align*}
  Fall 1: f_1\approx f_2  & & Fall 2: f_1\neq f_2.
\end{align*}
Bei Fall 1 verschwinden dadurch im Vergleich zu Fall 2 manche
Reflexe.
Die zu einem Reflex führenden Werte von hkl wurden hier mit Hilfe von
Python \cite[\python] berechent. Dabei ist nur $m_{sum}$ entscheidend
und von mehreren Ebenscharen mit identische $m_{sum}$ wird nur eine verwendet,
da alle den gleichen Beugungswinkel $\theta$ besitzen.
Es wird angenommen, dass alle Möglichen Reflexe bei der Messung auftreten.
Somit wird jedem Beugungswinkel \theta ein $m_{sum}$ zugewiesen.
Wobei der kleinste Beugungswinkel durch Beugung an der
Ebenschar mit der niedrigsten $m_sum$ entsteht und so weiter.
Die Tabelle \ref{tab:bcc} enthält bsw. die Zuordnung für eine bcc-Struktur.
Aus der Zuordnung von $m_{sum}$ und $\theta$
wird für jedem Messwert nach \eqref{eqn:Gitterkonst}
eine Gitterkonstante bestimmt. Der Fehler $\delta a$
auf die Gitterkonstante
wird nach den Gleichungen \eqref{eqn:sys1} und \eqref{eqn:sys2}
berechnet. Die berechneten Gitterkonstanten werden
gegen $\cos^2\left(\theta\right)$ aufgetragen und es wird
eine lineare Regression
der Datenpunkte mit Hilfe von Pyhton durchgeführt.
Die beste Gitterkonstante enspricht dann $a(\Si{90}{\degree})$
also dem y-Achsenabschitt der ermittelten Geraden.




Da die Gitterkonstanten $\symup{a}$ in der Theorie einen
festen Werte besitzt,
wird für größere Beugungswinkel $\theta$ auch

\begin{table}
  \centering
  \caption{Tabelle der Messwerte für die Kreisradien $r$ und die daraus nach \ref{eqn:winkel} resultienenden Winkel $\theta$
  und die von bcc zugeordneten Reflexe durch die Millerindices hkl und deren Quadratsumme. Ebenfalls aufgetragen sind die
  aus Gleichung \eqref{eqn:gitter} berechneten Gitterkonstanten $a$.}
  \label{tab:bcc}
\begin{tabular}{S S c c c S S S S}
\toprule
$r/\si{\milli\meter}$ & $\theta / \si{\degree}$&{$h$} & {$k$} & {$l$} & {$h^{2} + k^{2} + l^{2}$} & \multicolumn{2}{$a/\si{\angstrom}$} &  $\cos\left(\theta\right)^2$ \\
\midrule
4.2 	&	20.75	&	0	&	1	&	1	&	2 	&	3.08	\pm 	0.16	& 0.87   \\
6.0	  &	29.75	&	0	&	0	&	2	&	4 	&	3.11	\pm 	0.12	& 0.75   \\
7.4	  &	37.25	&	1	&	1	&	2	&	6 	&	3.12	\pm 	0.09	& 0.63   \\
8.8	  &	43.75	&	0	&	2	&	2	&	8 	&	3.15	\pm 	0.08	& 0.52   \\
10.0	&	50.25	&	0	&	1	&	3	&	10	&	3.17	\pm 	0.06	& 0.41   \\
11.4	&	57.25	&	2	&	2	&	2	&	12	&	3.18	\pm 	0.04	& 0.29   \\
13.0	&	65.25	&	1	&	2	&	3	&	14	&	3.18	\pm 	0.02	& 0.18   \\
15.3	&	76.49	&	1	&	1	&	4	&	18	&	3.36	\pm 	0.01	& 0.05   \\
\bottomrule
\end{tabular}
\end{table}



% plots
\begin{figure}[hhh]
  \centering
  \begin{subfigure}{.45\textwidth}
    \centering
    \includegraphics[width=\textwidth]{build/plot_ss1.pdf}
    \caption{plotss1}
    \label{subfig:cool11}
  \end{subfigure}
  \begin{subfigure}{.45\textwidth}
    \centering
    \includegraphics[width=\textwidth]{build/plot_fluor1.pdf}
    \caption{plotfluor1}
    \label{subfig:cool12}
  \end{subfigure}
  \begin{subfigure}{.45\textwidth}
    \centering
    \includegraphics[width=\textwidth]{build/plot_zb1.pdf}
    \caption{plotzb1}
    \label{subfig:cool13}
  \end{subfigure}
  \begin{subfigure}{.45\textwidth}
    \centering
    \includegraphics[width=\textwidth]{build/plot_cc1.pdf}
    \caption{plotcc1}
    \label{subfig:cool14}
  \end{subfigure}
  \caption{two subfigures}
  \label{fig:very cool1}
\end{figure}

\begin{figure}[hhh]
  \centering
  \begin{subfigure}{.45\textwidth}
    \centering
    \includegraphics[width=\textwidth]{build/plot_ss2.pdf}
    \caption{plotss2}
    \label{subfig:cool21}
  \end{subfigure}
  \begin{subfigure}{.45\textwidth}
    \centering
    \includegraphics[width=\linewidth]{build/plot_fluor2.pdf}
    \caption{plotfluor2}
    \label{subfig:cool22}
  \end{subfigure}
  \begin{subfigure}{.45\textwidth}
    \centering
    \includegraphics[width=\textwidth]{build/plot_zb2.pdf}
    \caption{plozb2}
    \label{subfig:cool23}
  \end{subfigure}
  \begin{subfigure}{.45\textwidth}
    \centering
    \includegraphics[width=\textwidth]{build/plot_cc2.pdf}
    \caption{plocc2}
    \label{subfig:cool24}
  \end{subfigure}
  \caption{two subfigures}
  \label{fig:very cool2}
\end{figure}
