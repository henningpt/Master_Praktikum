\section{Auswertung}
\label{sec:Auswertung}

Um mit Hilfe der Filmstreifenmethode die Struktur der Probe zu bestimmen, werden
die Radien der "Debye-Ringe", die sich durch die Bestrahlung auf dem Filmstreifen bilden,
vermessen. Durch die Proportionalität zwischen den Radien $r$ und den Beugungswinkeln $\theta$
\begin{align}
\theta=\frac{\SI{360}{\degree}\cdot r}{4 \pi R } \label{eqn:winkel}
\end{align}
mit dem Kameraradius $R$ wird der Beugungswinkel $\theta$ berechnet.
Die Tabelle \ref{tab:bcc} enthält die gemessenen Radien $r$
und die nach \ref{eqn:winkel} berechneten Winkel
für die Metallprobe und die Tabelle \ref{tab:ss1} die für die Salzprobe.
Es wird davon ausgegangen, dass die "Debye-Ringe" immer zu der ersten Ordnung gehören
und das nur kubische Gitterstrukturen beobachtet werden.
Somit lässt sich die Gleichung \ref{eqn:10} für auftretenden
Reflexe durch \eqref{eqn:5} zu
\begin{align}
\symup{a} = \frac{\lambda \sqrt{h^2 +k^2+l^2}}{2\sin{\theta}} \label{eqn:Gitterkonst}
\end{align}
umformen.
Um aus den berechneten Beugungswinkeln $\theta$ die Gitterkonstante a
zu bestimmen, muss somit auch die zum Reflex gehörige Millerindice-Summe
\begin{align*}
  m_{\mathrm{sum}}=h^{2} + k^{2} + l^{2}
\end{align*}
der Netzebenenschar hkl bekannt sein.
Die Information für welchen Werten von $hkl$ Refexe auftreten
enthält die Strukturamplitude $S$ nach Gleichung \eqref{eqn:Streu}.
Die Gleichung \eqref{eqn:Streu} wiederum benötigt die Basisvektoren
der Elementarzelle. Da die der Proben nicht bekannt sind, werden
Material typische Gitterstruktur verwendet und die Ergebnisse
untereinander verglichen. Metalle können typischerweise in
 bcc-, fcc- und Diamant-Struktur
vorliegen und Salz in Fluorit-, Zinkblende-,
Steinsalz-, Cäsiumchlorid-Struktur.
Durch die entsprechenden Basisvektoren
werden Werte für $hkl$ bestimmt, wo die Strukturamplitude
$S\neq0$ und somit ein Reflex auftritt.
Durch die Zweiatomigebasis bei den
Salzstrukturen ist es nötig zwischen zwei Fälle zu differenzieren,
da der Atomformfaktor $f$ in Gleichung
\eqref{eqn:Streu} nicht ausgeklammert werden kann.
\begin{align*}
  Fall 1: f_1\approx f_2  & & Fall 2: f_1\neq f_2.
\end{align*}
Bei Fall 1 verschwinden dadurch im Vergleich zu Fall 2 manche
Reflexe.
Die zu einem Reflex führenden Werte von $hkl$ wurden hier mit Hilfe von
\textit{Python3} berechnet. Dabei ist nur $m_{\mathrm{sum}}$ entscheidend
und von mehreren Ebenenscharen mit identische $m_{\mathrm{sum}}$ wird nur eine verwendet,
da alle den gleichen Beugungswinkel $\theta$ besitzen.
Es wird angenommen, dass alle möglichen Reflexe bei der Messung auftreten.
Somit wird jedem Beugungswinkel \theta ein $m_{\mathrm{sum}}$ zugewiesen.
Wobei der kleinste Beugungswinkel durch Beugung an der
Ebenenschar mit der niedrigsten $m_{\mathrm{sum}}$ entsteht und so weiter.
Die Tabelle \ref{tab:bcc} enthält bsw. die Zuordnung für eine bcc-Struktur.
Aus der Zuordnung von $m_{\mathrm{sum}}$ und $\theta$
wird für jedem Messwert nach \eqref{eqn:Gitterkonst}
eine Gitterkonstante bestimmt. Der Fehler $\Delta a$
auf die Gitterkonstante
wird nach den Gleichungen \eqref{eqn:sys1} und \eqref{eqn:sys2}
berechnet. Die berechneten Gitterkonstanten werden
gegen $\cos^2\left(\theta\right)$ aufgetragen und es wird
eine Fit der Datenpunkte der Form an die Funktion
\begin{align}
  f(x)= mx+b
\end{align}
mit Hilfe von \textit{Python3} durchgeführt.
Die beste Gitterkonstante entspricht dann $a(\SI{90}{\degree})$
also dem y-Achsenabschnitt der ermittelten Geraden somit dem Fitparameter $b$.
Die verwendeten Geräte besitzen folgende Maße:
\begin{align*}
  \rho &= \text{Probenradius} = \SI{1}{\milli\meter}\\
  R &= \text{Kameraradius} = \SI{57.3}{\milli\meter}\\
  F &= \text{Abstand Fokus-Probe} = \SI{130}{\milli\meter}.
\end{align*}
Die Wellenlängen der
$\mathrm{K}_{\alpha_1}-$
und $\mathrm{K}_{\alpha_2}$-Strahlung
werden für die Berechnungen gemittelt.
Für eine Cu-Anode ergibt sich somit
\begin{align*}
  \bar{\lambda}_{\mathrm{K}_\alpha} &= \SI{1,5417}{\angstrom}.
\end{align*}
Für größere Streuwinkel $\theta$ in der
Nähe von $\SI{90}{\degree}$
führt dies zu einer Ringaufspaltung
der Debye-Ring, die durch Mittelung der zwei Radien
Berücksichtigung findet. Zusätzlich
wird dem Streuwinkel \theta der Fehler
\begin{align*}
  \Delta\theta_{1,2} = \frac{\lambda_1 - \lambda_2}{\bar{\lambda}}\tan\theta\\
\intertext{mit}
 \lambda_1=\SI{1,54093}{\angstrom}\\
 \lambda_2=\SI{1,54478}{\angstrom}
\end{align*}
zugefügt.
\subsection{Metallprobe}
Die Metallprobe wird sowohl unter der Annahme einer
bcc-Struktur, siehe Tabelle \ref{tab:bcc} und Abbildung
 \ref{subfig:bcc},
einer fcc-Struktur, siehe Tabelle \ref{tab:fcc} und Abbildung \ref{subfig:fcc},
aus auch einer Diamant-Struktur, siehe Tabelle \ref{tab:dia} und Abbildung \ref{subfig:dia}
untersucht.


\begin{table}
  \centering
  \caption{Tabelle der Messwerte für die Kreisradien $r$ und die daraus nach \ref{eqn:winkel} resultierenden Winkel $\theta$
  und die von bcc zugeordneten Reflexe durch die Millerindices $hkl$ und deren Quadratsumme $m_{\mathrm{sum}}$.
  Ebenfalls aufgetragen sind die
  aus Gleichung \eqref{eqn:Gitterkonst} berechneten Gitterkonstanten $a$.}
  \label{tab:bcc}
\begin{tabular}{S S c c c c S[table-format=1.2]@{${}\pm{}$} S[table-format=1.2] S[table-format=1.3]}
\toprule
$r/\si{\milli\meter}$ & $\theta / \si{\degree}$ &$h$ & $k$ & $l$ & $m_{\mathrm{sum}}$ & \multicolumn{2}{c}{$a/\si{\angstrom}$}
&  {$\cos^2\left(\theta\right)$} \\
\midrule
4.2 	&	20.75	&	0	&	1	&	1	&	2 	&	3.08	&	0.16	&	0.874   \\
6.0 	&	29.75	&	0	&	0	&	2	&	4 	&	3.11	&	0.12	&	0.754   \\
7.4 	&	37.25	&	1	&	1	&	2	&	6 	&	3.12	&	0.09	&	0.634   \\
8.8 	&	43.75	&	0	&	2	&	2	&	8 	&	3.15	&	0.08	&	0.522   \\
10.0	&	50.25	&	0	&	1	&	3	&	10	&	3.17	&	0.06	&	0.409   \\
11.4	&	57.25	&	2	&	2	&	2	&	12	&	3.18	&	0.04	&	0.293   \\
13.0	&	65.25	&	1	&	2	&	3	&	14	&	3.18	&	0.02	&	0.175   \\
15.3	&	76.49	&	1	&	1	&	4	&	18	&	3.36	&	0.01	&	0.055   \\
\bottomrule
\end{tabular}
\end{table}



\begin{table}
\centering
\caption{Tabelle der Messwerte für die Kreisradien $r$ und die daraus nach \ref{eqn:winkel} resultierenden Winkel $\theta$
    und die von fcc zugeordneten Reflexe durch die Millerindices hkl und deren Quadratsumme $m_{\mathrm{sum}}$.
    Ebenfalls aufgetragen sind die
    aus Gleichung \eqref{eqn:Gitterkonst} berechneten Gitterkonstanten $a$.}
  \label{tab:fcc}
\begin{tabular}{S S c c c c S[table-format=1.2]@{${}\pm{}$} S[table-format=1.2] S[table-format=1.3]  }
\toprule
$r/\si{\milli\meter}$ & $\theta / \si{\degree}$ &$h$ & $k$ & $l$ & $m_{\mathrm{sum}}$ & \multicolumn{2}{c}{$a/\si{\angstrom}$}
&  {$\cos^2\left(\theta\right)$} \\
\midrule
4.2	&	20.75	&	1	&	1	&	1	&	3	&	3.77	&	0.19	&	0.874   \\
6.0	&	29.75	&	0	&	0	&	2	&	4	&	3.11	&	0.12	&	0.754   \\
7.4	&	37.25	&	0	&	2	&	2	&	8	&	3.60	&	0.11	&	0.634   \\
8.8	&	43.75	&	1	&	1	&	3	&	11	&	3.70	&	0.09	&	0.522   \\
10.0	&	50.25	&	2	&	2	&	2	&	12	&	3.47	&	0.06	&	0.409   \\
11.4	&	57.25	&	0	&	0	&	4	&	16	&	3.67	&	0.05	&	0.293   \\
13.0	&	65.25	&	1	&	3	&	3	&	19	&	3.70	&	0.03	&	0.175   \\
15.3	&	76.49	&	0	&	2	&	4	&	20	&	3.55	&	0.01	&	0.055   \\
\bottomrule
\end{tabular}
\end{table}




\begin{table}
\centering
\caption{Tabelle der Messwerte für die Kreisradien $r$ und die daraus nach \ref{eqn:winkel} resultierenden Winkel $\theta$
    und die von bcc zugeordneten Reflexe durch die Millerindices hkl und deren Quadratsumme $m_{\mathrm{sum}}$.
    Ebenfalls aufgetragen sind die
    aus Gleichung \eqref{eqn:Gitterkonst} berechneten Gitterkonstanten $a$.}
  \label{tab:dia}
\begin{tabular}{S S c c c c S[table-format=1.2]@{${}\pm{}$} S[table-format=1.2] S[table-format=1.3]  }
\toprule
$r/\si{\milli\meter}$ & $\theta / \si{\degree}$ &$h$ & $k$ & $l$ & $m_{\mathrm{sum}}$ & \multicolumn{2}{c}{$a/\si{\angstrom}$}
&  {$\cos^2\left(\theta\right)$} \\
\midrule
4.2 	&	20.75	&	1	&	1	&	1	&	3	  &	3.77	&	0.19	&	0.874   \\
6.0 	&	29.75	&	0	&	2	&	2	&	8  	&	4.39  &	0.17	&	0.754   \\
7.4 	&	37.25	&	1	&	1	&	3	&	11	&	4.22  &	0.13 	&	0.634   \\
8.8 	&	43.75	&	0	&	0	&	4	&	16  &	4.46  &	0.11	&	0.522   \\
10.0	&	50.25	&	1	&	3	&	3	&	19	&	4.37	&	0.08	&	0.409   \\
11.4	&	57.25	&	2	&	2	&	4	&	24	&	4.49	&	0.06	&	0.293   \\
13.0	&	65.25	&	3	&	3	&	3	&	27	&	4.41	&	0.03	&	0.175   \\
15.3	&	76.49	&	4	&	4	&	0	&	32	&	4.48	&	0.01	&	0.055   \\
\bottomrule
\end{tabular}
\end{table}


\begin{figure}[hhh]
  \centering
  \begin{subfigure}{0.7\textwidth}
    \centering
    \includegraphics[width=\textwidth]{build/plot_bcc.pdf}
    \caption{bcc-Struktur}
    \label{subfig:bcc}
  \end{subfigure}
  \begin{subfigure}{.49\textwidth}
    \centering
    \includegraphics[width=\textwidth]{build/plot_fcc.pdf}
    \caption{fcc-Struktur}
    \label{subfig:fcc}
  \end{subfigure}
  \begin{subfigure}{.49\textwidth}
    \centering
    \includegraphics[width=\textwidth]{build/plot_dia.pdf}
    \caption{Diamant-Struktur}
    \label{subfig:dia}
  \end{subfigure}
  \caption{Die bestimmten Gitterkonstante $a$ für verschiedene angenommene Strukturen gegen $\cos^2\theta$ aufgetragen.}
  \label{fig:metall}
\end{figure}
\FloatBarrier
Für die unterschiedlichen Strukturen für die metallische Probe ergeben sich folgende Gitterkonstanten $a$
aus dem Fitparameter $b$:
\begin{align*}
a_{\mathrm{bcc}}&=\SI{3.22(1)}{\angstrom}\\
a_{\mathrm{fcc}}&=\SI{3.7(2)}{\angstrom}\\
a_{\mathrm{dia}}&=\SI{4.7(2)}{\angstrom}
\end{align*}



\subsection{Salz-Probe}
Die Salz-Probe wird sowohl unter der Annahme
$f_1\approx f_2$  als auch $(f_1 \neq f_2)$ untersucht.
Dabei wird die Salz-Probe auf
Steinsalz-Struktur, siehe Tabelle \ref{tab:ss1}(\ref{tab:ss2}) und Abbildung  \ref{subfig:ss1} (\ref{subfig:ss2}),
Fluorit-Struktur, siehe Tabelle \ref{tab:fluo1}(\ref{tab:fluo2}) und Abbildung  \ref{subfig:fluo1} (\ref{subfig:fluo2}),
Zinkblende-Struktur, siehe Tabelle \ref{tab:zb1}(\ref{tab:zb2}) und Abbildung  \ref{subfig:zb1} (\ref{subfig:zb2})
und Cäsiumchlorid-Struktur, siehe Tabelle \ref{tab:cc1}(\ref{tab:cc2}) und Abbildung  \ref{subfig:cc1} (\ref{subfig:cc2})
überprüft.

\begin{table}
\centering
\caption{Tabelle der Messwerte für die Kreisradien $r$ und die daraus nach \ref{eqn:winkel} resultierenden Winkel $\theta$
    und die von der Steinsalz-Struktur für $f_1\approx f_2$ zugeordneten Reflexe durch die Millerindices hkl und deren Quadratsumme $m_{\mathrm{sum}}$.
    Ebenfalls aufgetragen sind die
    aus Gleichung \eqref{eqn:Gitterkonst} berechneten Gitterkonstanten $a$.}
  \label{tab:ss1}
\begin{tabular}{S S c c c c S[table-format=1.2]@{${}\pm{}$} S[table-format=1.2] S[table-format=1.3]  }
\toprule
$r/\si{\milli\meter}$ & $\theta / \si{\degree}$ &$h$ & $k$ & $l$ & $m_{\mathrm{sum}}$ & \multicolumn{2}{c}{$a/\si{\angstrom}$}
&  {$\cos^2\left(\theta\right)$} \\
\midrule
2.7	&	13.25	&	0	&	0	&	2	&	4	&	6.73	&	0.45	&	0.947   \\
3.0	&	14.75	&	0	&	2	&	2	&	8	&	8.56	&	0.54	&	0.935   \\
3.1	&	15.25	&	2	&	2	&	2	&	12	&	10.15	&	0.62	&	0.931   \\
4.4	&	21.75	&	0	&	0	&	4	&	16	&	8.32	&	0.41	&	0.863   \\
5.2	&	25.75	&	0	&	2	&	4	&	20	&	7.94	&	0.34	&	0.811   \\
5.4	&	26.75	&	2	&	2	&	4	&	24	&	8.39	&	0.35	&	0.797   \\
6.4	&	31.75	&	0	&	4	&	4	&	32	&	8.29	&	0.30	&	0.723   \\
7.0	&	34.75	&	2	&	4	&	4	&	36	&	8.11	&	0.27	&	0.675   \\
7.2	&	35.75	&	0	&	2	&	6	&	40	&	8.35	&	0.26	&	0.659   \\
8.0	&	39.75	&	2	&	2	&	6	&	44	&	8.00	&	0.22	&	0.591   \\
8.4	&	42.25	&	4	&	4	&	4	&	48	&	7.94	&	0.20	&	0.548   \\
9.4	&	47.25	&	0	&	4	&	6	&	52	&	7.57	&	0.16	&	0.461   \\
10.0	&	49.75	&	2	&	4	&	6	&	56	&	7.56	&	0.14	&	0.418   \\
10.1	&	50.75	&	0	&	0	&	8	&	64	&	7.96	&	0.14	&	0.400   \\
11.0	&	54.75	&	0	&	2	&	8	&	68	&	7.78	&	0.11	&	0.333   \\
11.5	&	57.75	&	2	&	2	&	8	&	72	&	7.73	&	0.09	&	0.285   \\
11.8	&	58.75	&	2	&	6	&	6	&	76	&	7.86	&	0.09	&	0.269   \\
13.0	&	65.25	&	0	&	4	&	8	&	80	&	7.59	&	0.06	&	0.175   \\
13.4	&	66.75	&	2	&	4	&	8	&	84	&	7.69	&	0.05	&	0.156   \\
13.6	&	67.75	&	4	&	6	&	6	&	88	&	7.81	&	0.05	&	0.143   \\
14.9	&	74.49	&	4	&	4	&	8	&	96	&	7.84	&	0.02	&	0.071   \\
16.4	&	81.74	&	0	&	6	&	8	&	100	&	7.79	&	0.01	&	0.021   \\
16.6	&	82.74	&	0	&	2	&	10	&	104	&	7.92	&	0.01	&	0.016   \\
\bottomrule
\end{tabular}
\end{table}

\begin{table}
\centering
\caption{Tabelle der Messwerte für die Kreisradien $r$ und die daraus nach \ref{eqn:winkel} resultierenden Winkel $\theta$
    und die von der Fluorit-Struktur für $f_1\approx f_2$ zugeordneten Reflexe durch die Millerindices hkl und deren Quadratsumme $m_{\mathrm{sum}}$.
    Ebenfalls aufgetragen sind die
    aus Gleichung \eqref{eqn:Gitterkonst} berechneten Gitterkonstanten $a$.}
  \label{tab:fluo1}
\begin{tabular}{S S c c c c S[table-format=1.2]@{${}\pm{}$} S[table-format=1.2] S[table-format=1.3]  }
\toprule
$r/\si{\milli\meter}$ & $\theta / \si{\degree}$ &$h$ & $k$ & $l$ & $m_{\mathrm{sum}}$ & \multicolumn{2}{c}{$a/\si{\angstrom}$}
&  {$\cos^2\left(\theta\right)$} \\
\midrule
2.7	&	13.25	&	0	&	1	&	1	&	2	&	4.76	&	0.32	&	0.947   \\
3.0	&	14.75	&	1	&	1	&	1	&	3	&	5.24	&	0.33	&	0.935   \\
3.1	&	15.25	&	0	&	0	&	2	&	4	&	5.86	&	0.36	&	0.931   \\
4.4	&	21.75	&	1	&	1	&	2	&	6	&	5.10	&	0.25	&	0.863   \\
5.2	&	25.75	&	0	&	2	&	2	&	8	&	5.02	&	0.22	&	0.811   \\
5.4	&	26.75	&	0	&	1	&	3	&	10	&	5.42	&	0.23	&	0.797   \\
6.4	&	31.75	&	1	&	1	&	3	&	11	&	4.86	&	0.17	&	0.723   \\
7.0	&	34.75	&	2	&	2	&	2	&	12	&	4.69	&	0.15	&	0.675   \\
7.2	&	35.75	&	1	&	2	&	3	&	14	&	4.94	&	0.16	&	0.659   \\
8.0	&	39.75	&	0	&	0	&	4	&	16	&	4.82	&	0.13	&	0.591   \\
8.4	&	42.25	&	1	&	1	&	4	&	18	&	4.86	&	0.12	&	0.548   \\
9.4	&	47.25	&	1	&	3	&	3	&	19	&	4.58	&	0.09	&	0.461   \\
10.0	&	49.75	&	0	&	2	&	4	&	20	&	4.52	&	0.08	&	0.418   \\
10.1	&	50.75	&	2	&	3	&	3	&	22	&	4.67	&	0.08	&	0.400   \\
11.0	&	54.75	&	2	&	2	&	4	&	24	&	4.62	&	0.07	&	0.333   \\
11.5	&	57.75	&	1	&	3	&	4	&	26	&	4.65	&	0.06	&	0.285   \\
11.8	&	58.75	&	3	&	3	&	3	&	27	&	4.69	&	0.05	&	0.269   \\
13.0	&	65.25	&	1	&	2	&	5	&	30	&	4.65	&	0.03	&	0.175   \\
13.4	&	66.75	&	0	&	4	&	4	&	32	&	4.75	&	0.03	&	0.156   \\
13.6	&	67.75	&	0	&	3	&	5	&	34	&	4.86	&	0.03	&	0.143   \\
14.9	&	74.49	&	1	&	3	&	5	&	35	&	4.73	&	0.01	&	0.071   \\
16.4	&	81.74	&	2	&	4	&	4	&	36	&	4.67	&	0.00	&	0.021   \\
16.6	&	82.74	&	2	&	3	&	5	&	38	&	4.79	&	0.00	&	0.016   \\
\bottomrule
\end{tabular}
\end{table}


\begin{table}
\centering
\caption{Tabelle der Messwerte für die Kreisradien $r$ und die daraus nach \ref{eqn:winkel} resultierenden Winkel $\theta$
    und die von der Zinkblende-Struktur für $f_1\approx f_2$ zugeordneten Reflexe durch die Millerindices hkl und deren Quadratsumme $m_{\mathrm{sum}}$.
    Ebenfalls aufgetragen sind die
    aus Gleichung \eqref{eqn:Gitterkonst} berechneten Gitterkonstanten $a$.}
  \label{tab:zb1}
\begin{tabular}{S S c c c c S[table-format=1.2]@{${}\pm{}$} S[table-format=1.2] S[table-format=1.3]  }
\toprule
$r/\si{\milli\meter}$ & $\theta / \si{\degree}$ &$h$ & $k$ & $l$ & $m_{\mathrm{sum}}$ & \multicolumn{2}{c}{$a/\si{\angstrom}$}
&  {$\cos^2\left(\theta\right)$} \\
\midrule
2.7	&	13.25	&	1	&	1	&	1	&	3	&	5.83	&	0.39	&	0.947   \\
3.0	&	14.75	&	0	&	2	&	2	&	8	&	8.56	&	0.54	&	0.935   \\
3.1	&	15.25	&	1	&	1	&	3	&	11	&	9.72	&	0.60	&	0.931   \\
4.4	&	21.75	&	0	&	0	&	4	&	16	&	8.32	&	0.41	&	0.863   \\
5.2	&	25.75	&	1	&	3	&	3	&	19	&	7.73	&	0.33	&	0.811   \\
5.4	&	26.75	&	2	&	2	&	4	&	24	&	8.39	&	0.35	&	0.797   \\
6.4	&	31.75	&	3	&	3	&	3	&	27	&	7.61	&	0.27	&	0.723   \\
7.0	&	34.75	&	0	&	4	&	4	&	32	&	7.65	&	0.25	&	0.675   \\
7.2	&	35.75	&	1	&	3	&	5	&	35	&	7.81	&	0.25	&	0.659   \\
8.0	&	39.75	&	0	&	2	&	6	&	40	&	7.62	&	0.21	&	0.591   \\
8.4	&	42.25	&	3	&	3	&	5	&	43	&	7.52	&	0.19	&	0.548   \\
9.4	&	47.25	&	4	&	4	&	4	&	48	&	7.27	&	0.15	&	0.461   \\
10.0	&	49.75	&	1	&	1	&	7	&	51	&	7.21	&	0.13	&	0.418   \\
10.1	&	50.75	&	2	&	4	&	6	&	56	&	7.45	&	0.13	&	0.400   \\
11.0	&	54.75	&	1	&	3	&	7	&	59	&	7.25	&	0.11	&	0.333   \\
11.5	&	57.75	&	0	&	0	&	8	&	64	&	7.29	&	0.09	&	0.285   \\
11.8	&	58.75	&	3	&	3	&	7	&	67	&	7.38	&	0.09	&	0.269   \\
13.0	&	65.25	&	2	&	2	&	8	&	72	&	7.20	&	0.05	&	0.175   \\
13.4	&	66.75	&	5	&	5	&	5	&	75	&	7.27	&	0.05	&	0.156   \\
13.6	&	67.75	&	0	&	4	&	8	&	80	&	7.45	&	0.04	&	0.143   \\
14.9	&	74.49	&	1	&	1	&	9	&	83	&	7.29	&	0.02	&	0.071   \\
16.4	&	81.74	&	4	&	6	&	6	&	88	&	7.31	&	0.01	&	0.021   \\
16.6	&	82.74	&	1	&	3	&	9	&	91	&	7.41	&	0.00	&	0.016   \\
\bottomrule
\end{tabular}
\end{table}



\begin{table}
\centering
\caption{Tabelle der Messwerte für die Kreisradien $r$ und die daraus nach \ref{eqn:winkel} resultierenden Winkel $\theta$
    und die von der Cäsiumchlorid-Struktur für $f_1\approx f_2$ zugeordneten Reflexe durch die Millerindices hkl und deren Quadratsumme $m_{\mathrm{sum}}$.
    Ebenfalls aufgetragen sind die
    aus Gleichung \eqref{eqn:Gitterkonst} berechneten Gitterkonstanten $a$.}
  \label{tab:cc1}
\begin{tabular}{S S c c c c S[table-format=1.2]@{${}\pm{}$} S[table-format=1.2] S[table-format=1.3]  }
\toprule
$r/\si{\milli\meter}$ & $\theta / \si{\degree}$ &$h$ & $k$ & $l$ & $m_{\mathrm{sum}}$ & \multicolumn{2}{c}{$a/\si{\angstrom}$}
&  {$\cos^2\left(\theta\right)$} \\
\midrule
2.7	&	13.25	&	0	&	1	&	1	&	2	&	4.76	&	0.32	&	0.947   \\
3.0	&	14.75	&	0	&	0	&	2	&	4	&	6.06	&	0.38	&	0.935   \\
3.1	&	15.25	&	1	&	1	&	2	&	6	&	7.18	&	0.44	&	0.931   \\
4.4	&	21.75	&	0	&	2	&	2	&	8	&	5.88	&	0.29	&	0.863   \\
5.2	&	25.75	&	0	&	1	&	3	&	10	&	5.61	&	0.24	&	0.811   \\
5.4	&	26.75	&	2	&	2	&	2	&	12	&	5.93	&	0.25	&	0.797   \\
6.4	&	31.75	&	1	&	2	&	3	&	14	&	5.48	&	0.20	&	0.723   \\
7.0	&	34.75	&	0	&	0	&	4	&	16	&	5.41	&	0.18	&	0.675   \\
7.2	&	35.75	&	1	&	1	&	4	&	18	&	5.60	&	0.18	&	0.659   \\
8.0	&	39.75	&	0	&	2	&	4	&	20	&	5.39	&	0.15	&	0.591   \\
8.4	&	42.25	&	2	&	3	&	3	&	22	&	5.38	&	0.14	&	0.548   \\
9.4	&	47.25	&	2	&	2	&	4	&	24	&	5.14	&	0.11	&	0.461   \\
10.0	&	49.75	&	1	&	3	&	4	&	26	&	5.15	&	0.10	&	0.418   \\
10.1	&	50.75	&	1	&	2	&	5	&	30	&	5.45	&	0.10	&	0.400   \\
11.0	&	54.75	&	0	&	4	&	4	&	32	&	5.34	&	0.08	&	0.333   \\
11.5	&	57.75	&	0	&	3	&	5	&	34	&	5.31	&	0.07	&	0.285   \\
11.8	&	58.75	&	2	&	4	&	4	&	36	&	5.41	&	0.06	&	0.269   \\
13.0	&	65.25	&	2	&	3	&	5	&	38	&	5.23	&	0.04	&	0.175   \\
13.4	&	66.75	&	0	&	2	&	6	&	40	&	5.31	&	0.03	&	0.156   \\
13.6	&	67.75	&	1	&	4	&	5	&	42	&	5.40	&	0.03	&	0.143   \\
14.9	&	74.49	&	2	&	2	&	6	&	44	&	5.31	&	0.02	&	0.071   \\
16.4	&	81.74	&	1	&	3	&	6	&	46	&	5.28	&	0.00	&	0.021   \\
16.6	&	82.74	&	4	&	4	&	4	&	48	&	5.38	&	0.00	&	0.016   \\
\bottomrule
\end{tabular}
\end{table}


\begin{table}
\centering
\caption{Tabelle der Messwerte für die Kreisradien $r$ und die daraus nach \ref{eqn:winkel} resultierenden Winkel $\theta$
    und die von der Steinsalz-Struktur für $f_1\neq f_2$ zugeordneten Reflexe durch die Millerindices hkl und deren Quadratsumme $m_{\mathrm{sum}}$.
    Ebenfalls aufgetragen sind die
    aus Gleichung \eqref{eqn:Gitterkonst} berechneten Gitterkonstanten $a$.}
  \label{tab:ss2}
\begin{tabular}{S S c c c c S[table-format=1.2]@{${}\pm{}$} S[table-format=1.2] S[table-format=1.3]  }
\toprule
$r/\si{\milli\meter}$ & $\theta / \si{\degree}$ &$h$ & $k$ & $l$ & $m_{\mathrm{sum}}$ & \multicolumn{2}{c}{$a/\si{\angstrom}$}
&  {$\cos^2\left(\theta\right)$} \\
\midrule
2.7	&	13.25	&	1	&	1	&	1	&	3	&	5.83	&	0.39	&	0.947   \\
3.0	&	14.75	&	0	&	0	&	2	&	4	&	6.06	&	0.38	&	0.935   \\
3.1	&	15.25	&	0	&	2	&	2	&	8	&	8.29	&	0.51	&	0.931   \\
4.4	&	21.75	&	1	&	1	&	3	&	11	&	6.90	&	0.34	&	0.863   \\
5.2	&	25.75	&	2	&	2	&	2	&	12	&	6.15	&	0.27	&	0.811   \\
5.4	&	26.75	&	0	&	0	&	4	&	16	&	6.85	&	0.29	&	0.797   \\
6.4	&	31.75	&	1	&	3	&	3	&	19	&	6.39	&	0.23	&	0.723   \\
7.0	&	34.75	&	0	&	2	&	4	&	20	&	6.05	&	0.20	&	0.675   \\
7.2	&	35.75	&	2	&	2	&	4	&	24	&	6.46	&	0.20	&	0.659   \\
8.0	&	39.75	&	3	&	3	&	3	&	27	&	6.26	&	0.17	&	0.591   \\
8.4	&	42.25	&	0	&	4	&	4	&	32	&	6.49	&	0.16	&	0.548   \\
9.4	&	47.25	&	1	&	3	&	5	&	35	&	6.21	&	0.13	&	0.461   \\
10.0	&	49.75	&	2	&	4	&	4	&	36	&	6.06	&	0.11	&	0.418   \\
10.1	&	50.75	&	0	&	2	&	6	&	40	&	6.30	&	0.11	&	0.400   \\
11.0	&	54.75	&	3	&	3	&	5	&	43	&	6.19	&	0.09	&	0.333   \\
11.5	&	57.75	&	2	&	2	&	6	&	44	&	6.05	&	0.07	&	0.285   \\
11.8	&	58.75	&	4	&	4	&	4	&	48	&	6.25	&	0.07	&	0.269   \\
13.0	&	65.25	&	1	&	1	&	7	&	51	&	6.06	&	0.04	&	0.175   \\
13.4	&	66.75	&	0	&	4	&	6	&	52	&	6.05	&	0.04	&	0.156   \\
13.6	&	67.75	&	2	&	4	&	6	&	56	&	6.23	&	0.04	&	0.143   \\
14.9	&	74.49	&	1	&	3	&	7	&	59	&	6.14	&	0.02	&	0.071   \\
16.4	&	81.74	&	0	&	0	&	8	&	64	&	6.23	&	0.01	&	0.021   \\
16.6	&	82.74	&	3	&	3	&	7	&	67	&	6.36	&	0.00	&	0.016   \\
\bottomrule
\end{tabular}
\end{table}

\begin{table}
\centering
\caption{Tabelle der Messwerte für die Kreisradien $r$ und die daraus nach \ref{eqn:winkel} resultierenden Winkel $\theta$
    und die von der Fluorit-Struktur für $f_1\neq f_2$ zugeordneten Reflexe durch die Millerindices hkl und deren Quadratsumme $m_{\mathrm{sum}}$.
    Ebenfalls aufgetragen sind die
    aus Gleichung \eqref{eqn:Gitterkonst} berechneten Gitterkonstanten $a$.}
  \label{tab:fluo2}
\begin{tabular}{S S c c c c S[table-format=1.2]@{${}\pm{}$} S[table-format=1.2] S[table-format=1.3]  }
\toprule
$r/\si{\milli\meter}$ & $\theta / \si{\degree}$ &$h$ & $k$ & $l$ & $m_{\mathrm{sum}}$ & \multicolumn{2}{c}{$a/\si{\angstrom}$}
&  {$\cos^2\left(\theta\right)$} \\
\midrule
2.7	&	13.25	&	0	&	1	&	1	&	2	&	4.76	&	0.32	&	0.947   \\
3.0	&	14.75	&	1	&	1	&	1	&	3	&	5.24	&	0.33	&	0.935   \\
3.1	&	15.25	&	0	&	0	&	2	&	4	&	5.86	&	0.36	&	0.931   \\
4.4	&	21.75	&	1	&	1	&	2	&	6	&	5.10	&	0.25	&	0.863   \\
5.2	&	25.75	&	0	&	2	&	2	&	8	&	5.02	&	0.22	&	0.811   \\
5.4	&	26.75	&	0	&	1	&	3	&	10	&	5.42	&	0.23	&	0.797   \\
6.4	&	31.75	&	1	&	1	&	3	&	11	&	4.86	&	0.17	&	0.723   \\
7.0	&	34.75	&	2	&	2	&	2	&	12	&	4.69	&	0.15	&	0.675   \\
7.2	&	35.75	&	1	&	2	&	3	&	14	&	4.94	&	0.16	&	0.659   \\
8.0	&	39.75	&	0	&	0	&	4	&	16	&	4.82	&	0.13	&	0.591   \\
8.4	&	42.25	&	1	&	1	&	4	&	18	&	4.86	&	0.12	&	0.548   \\
9.4	&	47.25	&	1	&	3	&	3	&	19	&	4.58	&	0.09	&	0.461   \\
10.0	&	49.75	&	0	&	2	&	4	&	20	&	4.52	&	0.08	&	0.418   \\
10.1	&	50.75	&	2	&	3	&	3	&	22	&	4.67	&	0.08	&	0.400   \\
11.0	&	54.75	&	2	&	2	&	4	&	24	&	4.62	&	0.07	&	0.333   \\
11.5	&	57.75	&	1	&	3	&	4	&	26	&	4.65	&	0.06	&	0.285   \\
11.8	&	58.75	&	3	&	3	&	3	&	27	&	4.69	&	0.05	&	0.269   \\
13.0	&	65.25	&	1	&	2	&	5	&	30	&	4.65	&	0.03	&	0.175   \\
13.4	&	66.75	&	0	&	4	&	4	&	32	&	4.75	&	0.03	&	0.156   \\
13.6	&	67.75	&	0	&	3	&	5	&	34	&	4.86	&	0.03	&	0.143   \\
14.9	&	74.49	&	1	&	3	&	5	&	35	&	4.73	&	0.01	&	0.071   \\
16.4	&	81.74	&	2	&	4	&	4	&	36	&	4.67	&	0.00	&	0.021   \\
16.6	&	82.74	&	2	&	3	&	5	&	38	&	4.79	&	0.00	&	0.016   \\
\bottomrule
\end{tabular}
\end{table}


\begin{table}
\centering
\caption{Tabelle der Messwerte für die Kreisradien $r$ und die daraus nach \ref{eqn:winkel} resultierenden Winkel $\theta$
    und die von der Zinkblende-Struktur für $f_1\neq f_2$ zugeordneten Reflexe durch die Millerindices hkl und deren Quadratsumme $m_{\mathrm{sum}}$.
    Ebenfalls aufgetragen sind die
    aus Gleichung \eqref{eqn:Gitterkonst} berechneten Gitterkonstanten $a$.}
  \label{tab:zb2}
\begin{tabular}{S S c c c c S[table-format=1.2]@{${}\pm{}$} S[table-format=1.2] S[table-format=1.3]  }
\toprule
$r/\si{\milli\meter}$ & $\theta / \si{\degree}$ &$h$ & $k$ & $l$ & $m_{\mathrm{sum}}$ & \multicolumn{2}{c}{$a/\si{\angstrom}$}
&  {$\cos^2\left(\theta\right)$} \\
\midrule
2.7	&	13.25	&	1	&	1	&	1	&	3	&	5.83	&	0.39	&	0.947   \\
3.0	&	14.75	&	0	&	0	&	2	&	4	&	6.06	&	0.38	&	0.935   \\
3.1	&	15.25	&	0	&	2	&	2	&	8	&	8.29	&	0.51	&	0.931   \\
4.4	&	21.75	&	1	&	1	&	3	&	11	&	6.90	&	0.34	&	0.863   \\
5.2	&	25.75	&	2	&	2	&	2	&	12	&	6.15	&	0.27	&	0.811   \\
5.4	&	26.75	&	0	&	0	&	4	&	16	&	6.85	&	0.29	&	0.797   \\
6.4	&	31.75	&	1	&	3	&	3	&	19	&	6.39	&	0.23	&	0.723   \\
7.0	&	34.75	&	0	&	2	&	4	&	20	&	6.05	&	0.20	&	0.675   \\
7.2	&	35.75	&	2	&	2	&	4	&	24	&	6.46	&	0.20	&	0.659   \\
8.0	&	39.75	&	3	&	3	&	3	&	27	&	6.26	&	0.17	&	0.591   \\
8.4	&	42.25	&	0	&	4	&	4	&	32	&	6.49	&	0.16	&	0.548   \\
9.4	&	47.25	&	1	&	3	&	5	&	35	&	6.21	&	0.13	&	0.461   \\
10.0	&	49.75	&	2	&	4	&	4	&	36	&	6.06	&	0.11	&	0.418   \\
10.1	&	50.75	&	0	&	2	&	6	&	40	&	6.30	&	0.11	&	0.400   \\
11.0	&	54.75	&	3	&	3	&	5	&	43	&	6.19	&	0.09	&	0.333   \\
11.5	&	57.75	&	2	&	2	&	6	&	44	&	6.05	&	0.07	&	0.285   \\
11.8	&	58.75	&	4	&	4	&	4	&	48	&	6.25	&	0.07	&	0.269   \\
13.0	&	65.25	&	1	&	1	&	7	&	51	&	6.06	&	0.04	&	0.175   \\
13.4	&	66.75	&	0	&	4	&	6	&	52	&	6.05	&	0.04	&	0.156   \\
13.6	&	67.75	&	2	&	4	&	6	&	56	&	6.23	&	0.04	&	0.143   \\
14.9	&	74.49	&	1	&	3	&	7	&	59	&	6.14	&	0.02	&	0.071   \\
16.4	&	81.74	&	0	&	0	&	8	&	64	&	6.23	&	0.01	&	0.021   \\
16.6	&	82.74	&	3	&	3	&	7	&	67	&	6.36	&	0.00	&	0.016   \\
\bottomrule
\end{tabular}
\end{table}



\begin{table}
\centering
\caption{Tabelle der Messwerte für die Kreisradien $r$ und die daraus nach \ref{eqn:winkel} resultierenden Winkel $\theta$
    und die von der Cäsiumchlorid-Struktur für $f_1\neq f_2$ zugeordneten Reflexe durch die Millerindices hkl und deren Quadratsumme $m_{\mathrm{sum}}$.
    Ebenfalls aufgetragen sind die
    aus Gleichung \eqref{eqn:Gitterkonst} berechneten Gitterkonstanten $a$.}
  \label{tab:cc2}
\begin{tabular}{S S c c c c S[table-format=1.2]@{${}\pm{}$} S[table-format=1.2] S[table-format=1.3]  }
\toprule
$r/\si{\milli\meter}$ & $\theta / \si{\degree}$ &$h$ & $k$ & $l$ & $m_{\mathrm{sum}}$ & \multicolumn{2}{c}{$a/\si{\angstrom}$}
&  {$\cos^2\left(\theta\right)$} \\
\midrule
2.7	&	13.25	&	0	&	0	&	1	&	1	&	3.36	&	0.22	&	0.947   \\
3.0	&	14.75	&	0	&	1	&	1	&	2	&	4.28	&	0.27	&	0.935   \\
3.1	&	15.25	&	1	&	1	&	1	&	3	&	5.08	&	0.31	&	0.931   \\
4.4	&	21.75	&	0	&	0	&	2	&	4	&	4.16	&	0.20	&	0.863   \\
5.2	&	25.75	&	0	&	1	&	2	&	5	&	3.97	&	0.17	&	0.811   \\
5.4	&	26.75	&	1	&	1	&	2	&	6	&	4.20	&	0.18	&	0.797   \\
6.4	&	31.75	&	0	&	2	&	2	&	8	&	4.14	&	0.15	&	0.723   \\
7.0	&	34.75	&	0	&	0	&	3	&	9	&	4.06	&	0.13	&	0.675   \\
7.2	&	35.75	&	0	&	1	&	3	&	10	&	4.17	&	0.13	&	0.659   \\
8.0	&	39.75	&	1	&	1	&	3	&	11	&	4.00	&	0.11	&	0.591   \\
8.4	&	42.25	&	2	&	2	&	2	&	12	&	3.97	&	0.10	&	0.548   \\
9.4	&	47.25	&	0	&	2	&	3	&	13	&	3.79	&	0.08	&	0.461   \\
10.0	&	49.75	&	1	&	2	&	3	&	14	&	3.78	&	0.07	&	0.418   \\
10.1	&	50.75	&	0	&	0	&	4	&	16	&	3.98	&	0.07	&	0.400   \\
11.0	&	54.75	&	0	&	1	&	4	&	17	&	3.89	&	0.06	&	0.333   \\
11.5	&	57.75	&	1	&	1	&	4	&	18	&	3.87	&	0.05	&	0.285   \\
11.8	&	58.75	&	1	&	3	&	3	&	19	&	3.93	&	0.05	&	0.269   \\
13.0	&	65.25	&	0	&	2	&	4	&	20	&	3.80	&	0.03	&	0.175   \\
13.4	&	66.75	&	1	&	2	&	4	&	21	&	3.84	&	0.03	&	0.156   \\
13.6	&	67.75	&	2	&	3	&	3	&	22	&	3.91	&	0.02	&	0.143   \\
14.9	&	74.49	&	2	&	2	&	4	&	24	&	3.92	&	0.01	&	0.071   \\
16.4	&	81.74	&	0	&	0	&	5	&	25	&	3.89	&	0.00	&	0.021   \\
16.6	&	82.74	&	1	&	3	&	4	&	26	&	3.96	&	0.00	&	0.016   \\
\bottomrule
\end{tabular}
\end{table}




% plots
\begin{figure}[hhh]
  \centering
  \begin{subfigure}{.45\textwidth}
    \centering
    \includegraphics[width=\textwidth]{build/plot_ss1.pdf}
    \caption{Steinsalz-Struktur}
    \label{subfig:ss1}
  \end{subfigure}
  \begin{subfigure}{.45\textwidth}
    \centering
    \includegraphics[width=\textwidth]{build/plot_fluor1.pdf}
    \caption{Fluorit-Struktur}
    \label{subfig:fluo1}
  \end{subfigure}
  \begin{subfigure}{.45\textwidth}
    \centering
    \includegraphics[width=\textwidth]{build/plot_zb1.pdf}
    \caption{Zinkblende-Struktur}
    \label{subfig:zb1}
  \end{subfigure}
  \begin{subfigure}{.45\textwidth}
    \centering
    \includegraphics[width=\textwidth]{build/plot_cc1.pdf}
    \caption{Cäsiumchlorid-Struktur}
    \label{subfig:cc1}
  \end{subfigure}
  \caption{Die bestimmten Gitterkonstante $a$ für verschiedene angenommene Strukturen gegen $\cos^2\theta$ aufgetragen, wobei
  $f_1\approx f_2$ angenommen wird.}
  \label{fig:salz1}
\end{figure}


\begin{figure}[hhh]
  \centering
  \begin{subfigure}{.45\textwidth}
    \centering
    \includegraphics[width=\textwidth]{build/plot_ss2.pdf}
    \caption{Steinsalz-Struktur}
    \label{subfig:ss2}
  \end{subfigure}
  \begin{subfigure}{.45\textwidth}
    \centering
    \includegraphics[width=\textwidth]{build/plot_fluor2.pdf}
    \caption{Fluorit-Struktur}
    \label{subfig:fluo2}
  \end{subfigure}
  \begin{subfigure}{.45\textwidth}
    \centering
    \includegraphics[width=\textwidth]{build/plot_zb2.pdf}
    \caption{Zinkblende-Struktur}
    \label{subfig:zb2}
  \end{subfigure}
  \begin{subfigure}{.45\textwidth}
    \centering
    \includegraphics[width=\textwidth]{build/plot_cc2.pdf}
    \caption{Cäsiumchlorid-Struktur}
    \label{subfig:cc2}
  \end{subfigure}
  \caption{Die bestimmten Gitterkonstante $a$ für verschiedene angenommene Strukturen gegen $\cos^2\theta$ aufgetragen, wobei
  $f_1\neq f_2$ angenommen wird.}
  \label{fig:salz2}
\end{figure}
\FloatBarrier
Für die unterschiedlichen Strukturen der Salz-Probe ergeben sich folgende Gitterkonstanten $a$
aus dem Fitparameter $b$:
\begin{align*}
\intertext{Fall 1:}
a_{\text{Steinsalz}}&=\SI{7.6(2)}{\angstrom}\\
a_{\text{Fluorit}}&=\SI{4.6(1)}{\angstrom}\\
a_{\text{Zinkblende}}&=\SI{7.1(3)}{\angstrom}\\
a_{\text{Cäsiumchlorid}} &=\SI{5.2(2)}{\angstrom}\\
\intertext{Fall 2:}
a_{\text{Steinsalz}}&=\SI{6.1(2)}{\angstrom}\\
a_{\text{Fluorit}}&=\SI{4.6(1)}{\angstrom}\\
a_{\text{Zinkblende}}&=\SI{6.1(2)}{\angstrom}\\
a_{\text{Cäsiumchlorid}} &=\SI{3.8(1)}{\angstrom}
\end{align*}
