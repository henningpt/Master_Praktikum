\section{Diskussion}
\label{sec:Diskussion}
Die folgenden Gitterkonstanten ergeben sich aus der Annahme einer
bestimmten Struktur des Kristalls.
\begin{table}
  \centering
\begin{tabular}{p{0.3\textwidth} p{0.3\textwidth} p{0.3\textwidth} }
{\begin{align*}
\intertext{Metall:}
a_\mathrm{bcc}&=\SI{3.22(1)}{\angstrom}\\
a_\mathrm{fcc}&=\SI{3.7(2)}{\angstrom}\\
a_\mathrm{dia}&=\SI{4.7(2)}{\angstrom}
\end{align*}}
&
{\begin{align*}
\intertext{Salz Fall 1:}
&a_{\text{Steinsalz}}&=\SI{7.6(2)}{\angstrom}\\
&a_{\text{Fluorit}}&=\SI{4.6(1)}{\angstrom}\\
&a_{\text{Zinkblende}}&=\SI{7.1(3)}{\angstrom}\\
&a_{\text{Cäsiumchlorid}} &=\SI{5.2(2)}{\angstrom}
\end{align*}}
&
{\begin{align*}
\intertext{Salz Fall 2:}
&a_{\text{Steinsalz}}&=\SI{6.1(2)}{\angstrom}\\
&a_{\text{Fluorit}}&=\SI{4.6(1)}{\angstrom}\\
&a_{\text{Zinkblende}}&=\SI{6.1(2)}{\angstrom}\\
&a_{\text{Cäsiumchlorid}} &=\SI{3.8(1)}{\angstrom}
\end{align*}}
\end{tabular}
\end{table}
Die Strukturen, die eine geringe Unsicherheit in der Gitterkonstante besitzen,
kommen als Struktur der Probe infrage.
Für die Metallprobe erfüllt dies die bcc-Struktur
und bei der Salzprobe kommen für $f_1 \approx f_2$
die Fluorit-Struktur und für $f_1 \neq f_2$
die Fluorit-Struktur als auch die Cäsiumchlorid-Struktur
infrage.
Werden die Gitterstruktur mit Literaturwerten
verglichen, ergibt sich für die Metallprobe das Element
Wolfram mit einer Gitterkonstante von
\begin{align*}
a_{\text{wolfram}}=\SI{3.16}{\angstrom}.\text{\cite{kittel}} \\
\intertext{Die Abweichung $A$ vom Messwert beträgt somit}
A=\SI{1.8(4)}{\percent}.
\end{align*}
Für die Salzprobe ergibt sich das Salz Cäsiumchlorid
mit der Cäsiumchlorid-Struktur
mit der Gitterkonstante
\begin{align*}
  a_{\text{CsCl}} = \SI{4.11}{\angstrom}
\intertext{als zutreffende Struktur. Die Abweichung $A$ vom Messwert beträgt somit}
  A=  \SI{8(3)}{\percent}.\text{\cite{cscl}}
\end{align*}
Die Abweichung zu dem Literaturwert für Wolfram
sind gering, dass hierbei von einer erfolgreichen Strukturbestimmung
auszugehen ist. Die vergleichsweise höher Abweichung zu dem Literaturwert
von Cäsiumchlorid liegt möglicherweise an falsch zugeordneten Reflex.
Dies zeigt, dass die Strukturbestimmung von Kristallen mit einatomiger Basis
weniger Fehleranfälliger ist, als die eines Kristalls mit mehratomiger Basis.
Zusammenfassend kann gesagt werden, dass die Debye-Scherrer-Methode
für eine Strukturbestimmung von Kristallen geeignet ist.
