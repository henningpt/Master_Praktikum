\section{Diskussion}
\label{sec:Diskussion}
Die folgenden Gitterkonstanten ergeben sich aus der Annahme einer
bestimmten Struktur des Kristalls.
\FloatBarrier
\begin{table}
  \centering
\begin{tabular}{p{0.3\textwidth} p{0.3\textwidth} p{0.3\textwidth} }
{\begin{align*}
\intertext{Metall:}
a_{\mathrm{bcc}}&=\SI{3.22(1)}{\angstrom}\\
a_{\mathrm{fcc}}&=\SI{3.7(2)}{\angstrom}\\
a_{\mathrm{dia}}&=\SI{4.7(2)}{\angstrom}
\end{align*}}
&
{\begin{align*}
\intertext{Salz Fall 1:}
&a_{\text{Steinsalz}}&=\SI{7.6(2)}{\angstrom}\\
&a_{\text{Fluorit}}&=\SI{4.6(1)}{\angstrom}\\
&a_{\text{Zinkblende}}&=\SI{7.1(3)}{\angstrom}\\
&a_{\text{Cäsiumchlorid}} &=\SI{5.2(2)}{\angstrom}
\end{align*}}
&
{\begin{align*}
\intertext{Salz Fall 2:}
&a_{\text{Steinsalz}}&=\SI{6.1(2)}{\angstrom}\\
&a_{\text{Fluorit}}&=\SI{4.6(1)}{\angstrom}\\
&a_{\text{Zinkblende}}&=\SI{6.1(2)}{\angstrom}\\
&a_{\text{Cäsiumchlorid}} &=\SI{3.8(1)}{\angstrom}
\end{align*}}
\end{tabular}
\end{table}
\FloatBarrier
Die Strukturen, die eine geringe Unsicherheit in der Gitterkonstante besitzen,
kommen als Struktur der Probe infrage.
Für die Metallprobe erfüllt dies die bcc-Struktur.
Da bei der Salzprobe keine Struktur besonders hervorsticht, sodass
eine Struktur klar zu geordnet werden kann, werden die berechneten
Gitterkonstanten auf ihre relative Abweichung $A$ zu Literaturwerten verglichen.
Es werden nur Literaturwerte berücksichtigt, die auch mit der Farbe der Proben übereinstimmen.
%  für $f_1 \approx f_2$
% die Fluorit-Struktur und für $f_1 \neq f_2$
% die Fluorit-Struktur als auch die Cäsiumchlorid-Struktur
% infrage.
Wird die Gitterstruktur der Metallprobe mit Literaturwerten
verglichen, ergibt sich für die Metallprobe das Element
Wolfram mit einer Gitterkonstante von
\begin{align*}
a_{\text{Wolfram}}=\SI{3.16}{\angstrom}.\text{\cite{kittel}} \\
\intertext{Die relative Abweichung $A$ vom Messwert beträgt somit}
A=\SI{1.8(4)}{\percent}.
\end{align*}

Für die Salzprobe treffen die Literaturwerte für den Fall $f_1\neq f_2$ von
\textbf{LiI} mit der Steinsalz-Struktur und von \textbf{4N$\text{H}_{\text{4}}$Cl} mit der Cäsiumchlorid-Struktur
am ehesten zu.
Die Gitterkonstante von \textbf{LiI} beträgt
\begin{align*}
  a_{\text{LiI}} = \SI{6.00}{\angstrom}.\text{\cite{LiI}}
\intertext{Die relative Abweichung $A$ vom Messwert beträgt somit}
  A=  \SI{1(3)}{\percent}.
\end{align*}
Die Gitterkonstante von \textbf{N$\text{H}_{\text{4}}$Cl} beträgt
\begin{align*}
  a_{\text{N$\text{H}_{\text{4}}$Cl}} = \SI{3.87}{\angstrom}.\text{\cite{kittel}}
\intertext{Die relative Abweichung $A$ vom Messwert beträgt somit}
  A=  \SI{2(3)}{\percent}.
\end{align*}
Das Salz \textbf{LiI} besitzt die geringste relative Abweichung zu dem
Messwert, deswegen kann von einer Steinsalz-Struktur ausgegangen werden.
Die Abweichung zu dem Literaturwert für Wolfram
sind gering, dass hierbei von einer erfolgreichen Strukturbestimmung
auszugehen ist. Die vergleichsweise höheren Abweichungen bei der
bestimmten Salzstruktur
liegen möglicherweise an falsch zugeordneten Reflexen.
Dies zeigt, dass die Strukturbestimmung von Kristallen sich als genauer erweist, wenn
der Kristall nur aus einem Element besteht.
Zusammenfassend kann gesagt werden, dass die Debye-Scherrer-Methode
für eine Strukturbestimmung von elementaren Kristallen geeignet ist, jedoch
für Kristalle aus mehreren Elementen müssen zur genauen Strukturbestimmung
andere Methoden hinzugezogen werden.
