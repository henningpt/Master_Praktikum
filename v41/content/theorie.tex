\section{Theorie}
\label{sec:Theorie}

\subsection{Zielsetzung}
\label{subsec:Zielsetzung}

\subsection{Kristallgitter}
\label{subsec:kristallstrukturen}

\subsection{Netzebenen}
\label{subsec:netzebenen}

\subsection{Beugung von Röntgenstrahlung an Kristallen}
\label{subsec:Beugung}
Kristallstrukturen können mit Hilfe von Beugung
elektromagnetischer Wellen an Elektron und Atomkernen untersucht werden.
Dabei eignet sich Röntgenstrahlung für die Untersuchung
, da die Wellenlänge dieser, die Größenordnung
von $\Si{1}{\Angstrom}$ besitzt.
Für die theorische Beschreibung werden Elektron und Atomkernen
als Hertzsche-Dipole betrachtet, die mit einem
Elektrischenfeld wechselwirken.
Somit ergibt sich die Intensität eines punktförmiges Streuzentrum mit
Ladung $ze_0$ zu
\begin{align}
  I(r,\theta,z)  \label{6}.
\end{align}
Durch die Endliche Ausdehnung der Elektronenhülle,
müssen jedoch Phasenunterschiede
berücksichtigt werden und es wird
der Atomformfaktor $f$ als Quotienten
\begin{align}
f=.
\end{align}
eigneführt.
Ebenfalls kann der Atomformfaktor $f$ als Fourier-Transformierte
der Ladungsverteilung $\rho(\vec{r})$
\begin{align}
f=\int
\end{align}
 angesehen werden.
Somit ergibt sich die Steuamplitude eines Atomes
\begin{align}
  I_a=
\end{align}
-erweiterung auf einheiets zelle
-erweiterung auf gitter
-> Braggbedingung an Netzebenen
-> Strukturamplitude S





\subsection{Debye-Scherrer Methode zur Strukturbestimmung}
\label{subsec:Methoden}

\cite{sample}
