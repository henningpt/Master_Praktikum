\section{Theorie}
\label{sec:Theorie}

\subsection{Zielsetzung}
\label{subsec:Zielsetzung}
Ziel dieses Versuchs ist es mit Hilfe der sogenannten
Debye-Scherrer-Methode die Struktur einer Kristallprobe
zu bestimmen und die Gitterkonstante zu messen.

\subsection{Allgemeines}
\label{subsec:allgemein}

Festkörper wie zum Beispiel einige Salze und Metalle
sind aus periodischen
Kristallstrukturen zusammengesetzt. Dabei sind im
Allgemeinen die einzelnen Kristallite zufällig
ausgerichtet. Wenn ein solcher Festkörper untersucht wird,
wird nur das statistische Mittel der Kristalle beobachtet.
Zur Untersuchung der eigentlichen Kristallstruktur
können somit nur einzelne Kristallite bzw.
Einkristalle verwendet werden.

\subsection{Kristallgitter}
\label{subsec:kristallstrukturen}
Um die periodischen Atomstrukturen eines Kristalls
zu beschreiben, wird ein Punktgitter verwendet.
Jedem Gitterpunkt wird eine festgelegte Atomgruppierung,
die sogenannte Basis, zugeordnet.
Es existieren theoretisch unendlich viele solcher Punktgitter.
Jedes wird durch drei sogenannte Translationsvektoren
$\vec{x}_{1}$, $\vec{x}_{2}$, $\vec{x}_{3}$ gekennzeichnet.
Durch Linearkombination dieser Vektoren lässt sich
jeder Gitterpunkt
\begin{align}
\label{eqn:1*}
\vec{t} = n_{1} \vec{x}_{1} + n_{2} \vec{x}_{2} + n_{3} \vec{x}_{3}
\end{align}
beschreiben.
Mit Hilfe des durch die drei Vektoren $\vec{x}_{i}$ aufgespannten
Parallelepipeds(siehe Abbildung \ref{fig:abb2}) lässt sich die
kleinstmögliche strukturbestimmende Einheit des Gitters,
die sogenannte Elementarzelle, erhalten.
Dabei wird zwischen primitiven und nichtprimitiven Einheitszellen
unterschieden. Eine primitive Einheitszelle schließt das minimal mögliche Volumen ein
und beinhaltet, unabhängig von ihrer Verschiebung, immer genau einen Punkt des
Gitters. Ist dies nicht der Fall handelt es sich um eine nichtprimitive Einheitszelle.\\
Die verschiedenen Punktgitter werden anhand ihrer Symmetrie
in 14 als \textit{Bravais-Gitter} bezeichnete Arten eingeteilt.
Diese lassen sich in 7 Gruppen zusammenfassen.

\begin{figure}[hhh]
\centering
\includegraphics[width=0.5\textwidth]{abbildungen/epiped.png}
\caption{Durch Translationsvektoren aufgespanntes Parallelepiped.\cite{sample}}
\label{fig:abb2}
\end{figure}

\subsubsection{Kubische Kristallstrukturen}
\label{subsubsec:kubische_gitter}

Die Gitterstruktur von vielen in der Natur vorkommenden Elementen,
sind häufig gegeben durch die sogenannten kubischen Gitter.
Bei kubischen Gittern besitzen die drei Translationsvektoren
alle die selbe Länge und stehen jeweil senkrecht aufeinander.\\
Es wird zwischen \textit{kubisch-primitiven},
\textit{kubisch-flächenzentrierten}(\textbf{bcc}) und \\
\textit{kubisch-raumzentrierten Gittern}(\textbf{fcc}) unterschieden.\\
Das \textit{kubisch-primitive Gitter}
kennzeichnet sich dadurch,
dass die Atome der Einheitszelle auf den Ecken
eines Würfels platziert werden können.\\
Das \textit{kubisch raumzentrierte Gitter}(siehe Abbildung \ref{fig:bcc})
gleicht dem primitiven, allerdings befindet sich noch
zusätzlich ein Atom im Mittleren des Würfels.
Somit besteht in diesem Fall die Basis aus zwei Atomen.
Die Atome in der Einheitszelle sind durch
\begin{align}
\label{eqn:2*}
\begin{pmatrix}
0, 0, 0;
\end{pmatrix} \
\begin{pmatrix}
\frac{1}{2}, \frac{1}{2}, \frac{1}{2};
\end{pmatrix}
\end{align}
gegeben.\\

\begin{figure}[hhh]
\centering
\includegraphics[width=0.5\textwidth]{abbildungen/bcc.png}
\caption{Einheitszelle des kubisch-raumzentrierten Gitters.\cite{sample}}
\label{fig:bcc}
\end{figure}

Das \textit{kubisch flächenzentrierte Gitter} gleicht dem
\textit{kubisch-primitiven}, hat
allerdings zusätzlich jeweils auf den Mitten der Würfelflächen Atome.
Die Atome in der Einheitszelle sind durch
\begin{align}
\label{eqn:3*}
\begin{pmatrix}
0, 0, 0;
\end{pmatrix}\
\begin{pmatrix}
\frac{1}{2}, \frac{1}{2}, 0;
\end{pmatrix}\
\begin{pmatrix}
\frac{1}{2}, 0, \frac{1}{2};
\end{pmatrix}\
\begin{pmatrix}
0, \frac{1}{2}, \frac{1}{2};
\end{pmatrix}
\end{align}
gegeben.

\begin{figure}[hhh]
\centering
\includegraphics[width=0.5\textwidth]{abbildungen/fcc.png}
\caption{Einheitszelle des kubisch-flächenzentrierten Gitters.\cite{sample}}
\label{fig:fcc}
\end{figure}

Aus zwei \textit{kubisch-flächenzentrierten Gittern}, die
um eine viertel Raumdiagonale zu einander verschoben sind, ergibt sich die sogenannte
\textit{Diamant-Struktur}.
Die zugehörigen Atome in der Einheitszelle sind gegeben durch die
Vektoren des \textit{kubisch-flächenzentrierten Gitters}
und zusätzlich durch:
\begin{align}
\label{eqn:4*}
\begin{pmatrix}
\frac{1}{4}, \frac{1}{4}, \frac{1}{4}
\end{pmatrix}\
\begin{pmatrix}
\frac{3}{4}, \frac{3}{4}, \frac{1}{4}
\end{pmatrix}\
\begin{pmatrix}
\frac{3}{4}, \frac{1}{4}, \frac{3}{4}
\end{pmatrix}\
\begin{pmatrix}
\frac{1}{4}, \frac{3}{4}, \frac{3}{4}
\end{pmatrix}
\end{align}

\subsection{Kristallstrukturen aus zwei Elementen}
\label{subsec:2atome}
Die bisher diskutierten Kristallstrukturen sind alle aus nur
einem Element zusammengesetzt. Im Folgenden wird auf Strukturen
eingegangen, die aus zwei verschiedenen Elementen bestehen.\\
Die sogeannte \textit{Zinkblende-Struktur} entsteht aus der
in \ref{subsubsec:kubische_gitter}
aufgeführten Diamantstruktur, indem an den Stellen der zum
kubisch-flächenzentrierten verschiedenen Vektoren der Einheitszelle
ein anderer Element auftritt.\\
Bei der \textit{Steinsalz-Struktur}
sind die verschiedenen Elementen
jeweils in zwei kubisch-flächenzentrierten Gittern angeordnet, die um eine
halbe Raumdiagonale zueinander verschoben sind. Die Atompositionen
in der Einheitszelle sind:
\begin{align}
\label{eqn:5*}
\text{Atom }(1) :&
\begin{pmatrix}
0, 0, 0
\end{pmatrix}\
\begin{pmatrix}
\frac{1}{2}, \frac{1}{2}, 0
\end{pmatrix}\
\begin{pmatrix}
\frac{1}{2}, 0, \frac{1}{2}
\end{pmatrix}\
\begin{pmatrix}
0, \frac{1}{2}, \frac{1}{2}
\end{pmatrix}\\
\label{eqn:6*}
\text{Atom }(2) :&
\begin{pmatrix}
0, 0, 0
\end{pmatrix}\
\begin{pmatrix}
1, 1, \frac{1}{2}
\end{pmatrix}\
\begin{pmatrix}
1, \frac{1}{2}, 1
\end{pmatrix}\
\begin{pmatrix}
\frac{1}{2}, 1, 1
\end{pmatrix}
\end{align}
Die sogenannte \textit{Cäsium-Chlorid-Struktur}
ensteht, wenn zwei \textit{kubisch-primitive Gitter}, mit jeweils
unterschiedlichen Atomen, um eine halbe Raumdiagonale zueinander
verschoben werden. Die Atompositionen in der Einheitszelle
sind gegeben durch:
\begin{align}
\label{eqn:7*}
\text{Atom }(1) :&
\begin{pmatrix}
0, 0, 0
\end{pmatrix}\\
\label{eqn:8*}
\text{Atom }(1) :&
\begin{pmatrix}
\frac{1}{2}, \frac{1}{2}, \frac{1}{2}
\end{pmatrix}
\end{align}
Als weitere relevante Struktur ist noch die
\textit{Fluorit-Struktur} zu nennen.
Ein Element(\textbf{1}) besetzt die Plätze des
\textit{kubisch-flächenzentrierten Gitters},
das Andere (\textbf{2}) die folgenden
Positionen:
\begin{align}
\label{eqn:9*}
\text{Atom }(2) :&
\begin{pmatrix}
\frac{1}{4}, \frac{1}{4}, \frac{1}{4}
\end{pmatrix}\
\begin{pmatrix}
\frac{3}{4}, \frac{3}{4}, \frac{1}{4}
\end{pmatrix}\
\begin{pmatrix}
\frac{1}{4}, \frac{3}{4}, \frac{3}{4}
\end{pmatrix}\
\begin{pmatrix}
\frac{3}{4}, \frac{3}{4}, \frac{3}{4}
\end{pmatrix}\\
&\begin{pmatrix}
\frac{1}{4}, \frac{1}{4}, \frac{3}{4}
\end{pmatrix}\
\begin{pmatrix}
\frac{1}{4}, \frac{3}{4}, \frac{1}{4}
\end{pmatrix}\
\begin{pmatrix}
\frac{3}{4}, \frac{1}{4}, \frac{1}{4}
\end{pmatrix}\
\end{align}

\subsection{Netzebenen}
\label{subsec:netzebenen}
Bei der Beugung von Röntgenstrahlung an Kristallen,
auf die im folgenden Abschnitt \ref{subsec:Beugung}
eingegangen wird, spielen die sogenannten \textit{Netzebenen}
eine eine wichtige Rolle. Dabei handelt es sich
um Ebenen, die Atomschwerpunkte beinhalten.\\
Zu jeder Netzebene existieren, bedingt durch die
Periodizität von Kristallstrukturen, weitere zueinander
äquidistant verschobene Parallelebenen.
Es lassen sich also Scharen von Netzebenen betrachten,
deren Lage üblicherweise durch ein Zahlentripel $(h \ k \ l)$,
den \textit{Miller-Indizes}, charakterisiert wird.\\
Um die \textit{Miller-Indizes} zu erhalten, wird die
Ebene einer Ebenenschar betrachtet, welche den geringsten
Abstand zum Ursprung besitzt. Die reziproken Werte
der Schnittpunkte dieser Ebene mit den Achsen des
Koordinatensystems der Einheitszelle ergeben im
dreidimensionalen Raum das gesuchte Zahlentripel.\\
Um die typische Notation mit drei ganzen Zahlen zu erhalten
wird das Zahlentripel noch mit einer geeigneten Zahl
multipliziert. In Abbildung \ref{fig:miller_ind}
ist die Beschreibung für die Netzebenenschar mit den
Indizes $(1 \ 4 \ 6)$ veranschaulicht.
Liegt kein Schnittpunkt mit einer
Achse vor, so wird der Millerindex $0$ gewählt.

\begin{figure}[hhh]
\centering
\includegraphics[width=0.5\textwidth]{abbildungen/miller_ind.png}
\caption{Veranschaulichung einer Netzebene mit Miller-Indizes (1 4 6).\cite{sample}}
\label{fig:miller_ind}
\end{figure}

Wie bereits erwähnt, haben benachbarte Netzebenen immer
den selben Abstand. Dieser Netzebenenabstand $d$ wird bei der
Strukturanalyse benötigt. In Abbildung \ref{fig:miller_abstand} wird
die Berechnung von $d$ veranschaulicht. Der Vektor, dessen Betrag $d$ ergibt,
muss einerseits orthogonal auf der eingezeichneten Netzebene und andererseits orthogonal
auf einer Benachbarten. Aus Überlegungen zu den in Abbildung \ref{fig:miller_abstand}
eingezeichneten rechtwinkligen Dreiecken unter der Bedingung
\begin{align}
\label{eqn:4}
\cos\alpha^2 + \cos\beta^2 + \cos\gamma^2 = 1
\end{align}
lässt sich für kubische Kristallstrukturen der Netzebenenabstand
durch
\begin{align}
\label{eqn:5}
d = \frac{a}{\sqrt{h^2 + k^2 + l^2}}
\end{align}
berechnen.

\begin{figure}[hhh]
\centering
\includegraphics[width=0.5\textwidth]{abbildungen/miller_dist.png}
\caption{Skizze zur Veranschaulichung der Berechnung des Netzebenenabstands.\cite{sample}}
\label{fig:miller_abstand}
\end{figure}

\FloatBarrier

\subsection{Beugung von Röntgenstrahlung an Kristallen}
\label{subsec:Beugung}
Kristallstrukturen können mit Hilfe von Beugung
elektromagnetischer Wellen an Elektron und Atomkernen untersucht werden.
Dabei eignet sich Röntgenstrahlung
für die Untersuchung, da diese
Wellenlängen besitzt, die in der
Größenordnung der Gitterkonstanten liegen.
Für die theoretische Beschreibung werden Elektron
und Atomkernen
als Hertzsche-Dipole betrachtet, die durch die
Röntgenstrahlung in Schwingung versetzt
werden und deshalb selbst
Strahlung emittieren.
Für die Intensität eines punktförmiges Streuzentrum mit
Ladung $ze_0$ ergibt sich
\begin{align}
I(r,\theta,z) = I_0\left(\frac{\mu_0 (ze_0)^2}{4\pi zm_0}\right)\frac{1}{r^2}\frac{1+\cos^2 2\theta}{2}=z^2 I_e \label{6}.
\end{align}
Da die Intensität invers
quadratisch zu der Masse $m_0$ des Streuzentrums ist,
wird im folgenden die Atomkerne vernachlässigt und
nur die Elektronenhülle des Atroms betrachtet.
Durch die endliche Ausdehnung der Elektronenhülle,
müssen jedoch Phasenunterschiede
\begin{align*}
\Delta\phi=2\pi\frac{\Delta s}{\lambda}= 2\pi\vec{r}\left(\vec{k}-\vec{k}_0\right)
\end{align*}
zwischen den gebeugten
Strahlen berücksichtigt werden und es wird
der Atomformfaktor $f$ als Quotient
von am Atom gestreuter Intensität $I_a$ und $I_e$ der
Intensität eines Einzelelektrons
\begin{align}
f^2=\frac{I_a}{I_e}
\end{align}
eingeführt.

Ebenfalls kann der Atomformfaktor $f$ als Fourier-Transformierte
der Ladungsverteilung $\rho(\vec{r})$
\begin{align}
f=\int_{\mathrm{Hülle}}
\symup{e}^{-\symup{i}\pi\Delta\phi}
\rho\left( \vec{r}\right) \symup{d}^3 r
= \int_{\mathrm{Hülle}}
\symup{e}^{-2\pi\symup{i}
\left(\vec{k}-\vec{k}_0\right)}
\rho\left( \vec{r}\right) \symup{d}^3 r
\end{align}
angesehen werden.
Somit ergibt sich die Streuamplitude
\begin{align}
I_{\mathrm{a}}=f^2\left(z,\frac{\sin\theta}{\lambda}\right)I_{\mathrm{e}}\left(r,\theta\right)
\end{align}
eines Atome.
Die Streuamplitude $S$ einer Elementarzelle resultiert aus
phasenrichtiger Aufsummation der
einzelnen Atomstreuamplituden, wobei unterschiedliche
Atome an den Orten $\vec{r_j}$
andere Formfaktoren $f_j$ besitzen.
Somit folgt für die Streuamplitude oder auch Strukturamplitude genannt
% \begin{align}
% S=\sum_j \symup{e}^{-2\pi \symup{i} \vec{r}_j\left(\vec{k}-\vec{k}_0\right)} I_{\mathrm{e}}.
% \end{align}
Die Streuung findet nicht nur an
einer Elementarzelle statt, sondern an
Ebenenscharen im Kristall statt.
Wird dies berücksichtigt, ergibt sich die
Braggsche Bedingung
\begin{align}
n\lambda= 2\symup{d}\sin{\theta} \ \ \ \ n=1,2\dots \label{eqn:10}
\end{align}
mit dem Ebenenabstand $d$ und Ein- und Ausfallwinkel $\theta$.
Jeder auftretende Reflex erfüllt diese Bedingung.
Durch die dazu äquivalente Form
\begin{align}
\vec{k}-\vec{k}_0=\frac{n}{|\vec{d}|}=\vec{g}
\intertext{mit dem reziproken Gittervektor\footnote{Die Basisvektoren und Reziprokenbasisvektoren erfüllen die Relation $A_i*a_j=\delta_{ij}$}}
\vec{g}(hkl)=h\vec{A}+k\vec{B}+l\vec{C}
\end{align}
folgt für die Streuamplitude der $(hkl)$-Ebene
\begin{align}
S=\sum_j \symup{e}^{-2\pi \symup{i}(x_jh+y_jk+z_jl)} I_{\mathrm{e}}. \label{eqn:Streu}
\end{align}
Sind sowohl die Atomformfaktor $f_j$ und
die Positionen $\vec{r}_j$ der
Basis bekannt, ist durch diese
Gleichung festgelegt,
ob ein Reflex für die $(hkl)$-Ebene
beobachtet wird $S\neq0$ oder nicht $S=0$.
% erweiterung auf einheiets zelle
% -erweiterung auf gitter
% -&gt; Braggbedingung an Netzebenen
% -&gt; Strukturamplitude S
%
\subsection{Debye-Scherrer Methode}
\label{subsec:Methoden}
Um die Kristallstruktur eines Probematerials
zu ermitteln, wird bei der Debye-Scherrer Methode
das Probematerial mit monochromatischer
Röntgenstrahlung bestrahlt.
Hierbei ist die Probe kein
Einkristall, sondern eine
kristalline, fein pulverisierte Probe
in der die Mikrokristallorientierungen statistisch
über den ganzen Raumwinkel verteilt sind.
Dies erhöht die Wahrscheinlichkeit, dass
einige Mikrokristalle in Reflexionsstellung
befinden und ein Bragg-Reflex beobachtet wird.
Mit Hilfe eines
Zählrohr-Goniometers oder
eines Filmstreifens wird der Beugungswinkel $\theta$
bestimmt. Durch systematisches Probieren wird
jedem Beugungswinkel $\theta$ eine Netzebenenschar
${(h k l)}$ zugeordnet und somit versucht
eine zu den Reflexen passende Kristallstruktur und
die Gitterkonstante $a$
zu ermitteln.
