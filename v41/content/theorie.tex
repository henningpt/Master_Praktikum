\section{Theorie}
\label{sec:Theorie}

\subsection{Zielsetzung}
\label{subsec:Zielsetzung}
Ziel dieses Versuchs ist es, mit Hilfe der sogenannten
Debye-Scherrer-Methode die Struktur einer Kristallprobe
zu bestimmen und die Gitterkonstante zu messen.

\subsection{Allgemines}
\label{subsec:allgemein}

Festkörper wie zum Beispiel Salze und Metalle
sind aus periodischen
Kristallstrukturen zusamengesetzt. Dabei sind im
Allgemeinen die einzelenen Kristallite zufällig
ausgerichtet. Wenn ein solcher Festkörper untersucht wird,
wird nur das statistische Mittel der Kristalle beobachtet.
Zur Untersuchung der Eigentlichen Kristallstruktur
können somit nur einzelne Kleinkristalle bzw.
Einkristalle(???) verwendet werden.


\subsection{Kristallgitter}
\label{subsec:kristallstrukturen}
Um die Periodischen Atomstrukturen eines Kristalls
zu beschreiben, wird ein Punktgitter verwendet.
Jedem Gitterpunkt wird eine festgelegte Atomgruppierung,
die sogenannte Basis, zugeordnet.
Es existieren theoretisch unendlich viele solcher Punktgitter.
Jedes wird durch drei sogenannte Translationsvektoren
$\vec{x}_{1}$, $\vec{x}_{2}$, $\vec{x}_{3}$ gekennzeichnet.
Durch Linearkombination dieser Vektoren lässt sich
jeder Gitterpunkt
\begin{align}
  \label{eqn:1}
  \vec{t} = n_{1} \vec{x}_{1} + n_{2} \vec{x}_{2} + n_{3} \vec{x}_{3}
\end{align}
beschreiben.
Mit Hilfe des durch die drei Vektoren $x_{i}$ aufgespannten
Parallelepipeds(siehe Abbildung \ref{fig:abb2}) lässt sich die
kleinstmögliche strukturbestimmende Einheit des Gitters,
die sogenannte Elementarzelle, erhalten.
Dabei wird zwischen primitiven und nichtprimitiven Einheitszellen
unterschieden. Im Fall einer primitiven Einheitszelle,
liegen auf den Eckpunkten des Parallelepipeds Atome.
Ist dies nicht der Fall, handelt es sich um eine
nichtprimitive Einheitszelle.\\ \\
Die verschiedenen Punktgitter werden anhand ihrer Symmetrie
in 14 als Bravais-Gitter bezeichnete Arten eingeteilt.
Diese lassen sich in 7 Gruppen zusammenfassen.\\ \\
\subsubsection{Kubische Kristallstrukturen}
\label{subsubsec:kubische_gitter}

Die Gitterstruktur von Metallen und Salzen, sind häufig gegeben
durch die sogenannten Kubischen Gitter.
Bei kubischen Gittern sind besitzen die drei Translationsvektoren
alle die selbe Länge und stehen jeweil senkrecht aufeinander.\\
Es wird zwischen kubisch-primitiven,
kubisch-flächenzentrierten und
kubisch-raumzentrierten Gittern unterschieden. \\ \\
Das kubisch-primitive Gitter(siehe Abbildung \ref{fig:bcc})
kennzeichnet sich dadurch,
das die Atome der Einheitszelle auf den Ecken
eines Würfels platziert sind. Somit kann als Basisvektor trivial
der Nullvektor angegeben werden.\\ \\
Das kubisch raumzentrierte Gitter(siehe Abbildung \ref{fig:raumzentriert})
gleicht dem primitiven, allerdings befindet sich noch
zusätzlich ein Atom im Mittleren des Würfels.
Somit besteht in diesem Fall die Basis aus zwei Atomen.
Diese Basisvektoren sind durch
\begin{align}
  \label{eqn:2}
  \begin{pmatrix}
    0, 0, 0;
  \end{pmatrix}\\
  \begin{pmatrix}
    \frac{1}{2}, \frac{1}{2}, \frac{1}{2};
  \end{pmatrix}
\end{align}
gegeben.\\ \\
Das kubisch flächenzentrierte Gitter gleicht dem kubisch-primitiven, hat
allerdings zusätzlich jeweils auf den Mitten der Würfelflächen Atome.
Die Basisvektoren sind durch
\begin{align}
   \label{eqn:3}
   \begin{pmatrix}
     0, 0, 0;
   \end{pmatrix}\\
   \begin{pmatrix}
     \frac{1}{2}, \frac{1}{2}, 0;
   \end{pmatrix}\\
   \begin{pmatrix}
     \frac{1}{2}, 0, \frac{1}{2};
     \end{pmatrix}\\
     \begin{pmatrix}
       0, \frac{1}{2}, \frac{1}{2};
     \end{pmatrix}
\end{align}
gegeben.

Durch Verschiebung des kubisch-flächenzentrierten Gitters
um eine viertel Raumdiagonale ergibt sich das sogenannte
Diamantgitter(siehe Abbildung \ref{fig:diamant}).
Die zugehörigen Basisvektoren sind gegeben durch
\begin{alig}
  \label{eqn:4}
  \begin{pmatrix}
    
  \end{pmatrix}
\end{alig}
\subsection{Netzebenen}
\label{subsec:netzebenen}

\subsection{Beugung von Röntgenstrahlung an Kristallen}
\label{subsec:Beugung}
Kristallstrukturen können mit Hilfe von Beugung
elektromagnetischer Wellen an Elektron und Atomkernen untersucht werden.
Dabei eignet sich Röntgenstrahlung für die Untersuchung
, da die Wellenlänge dieser, die Größenordnung
% von \Si{1}{\angstrom}
 besitzt.
Für die theorische Beschreibung werden Elektron und Atomkernen
als Hertzsche-Dipole betrachtet, die mit einem
Elektrischenfeld wechselwirken.
Somit ergibt sich die Intensität eines punktförmiges Streuzentrum mit
Ladung $ze_0$ zu
\begin{align}
  I(r,\theta,z)  \label{6}.
\end{align}
Durch die endliche Ausdehnung der Elektronenhülle,
müssen jedoch Phasenunterschiede
berücksichtigt werden und es wird
der Atomformfaktor $f$ als Quotienten
\begin{align}
f=.ate
\end{align}
eingeführt.
Ebenfalls kann der Atomformfaktor $f$ als Fourier-Transformierte
der Ladungsverteilung $\rho(\vec{r})$
\begin{align}
f=\int
\end{align}
angesehen werden.
Somit ergibt sich die Steuamplitude
\begin{align}
  I_\mathrm{a}=f^2\left(z,\frac{\sin\theta}{\lambda}\right)I_\mathrm{e}\left(r,\theta\right)
\end{align}
eines Atomes.
Die Streuamplitude $A$ einer Elementarzelle resultiert aus
phasenrichtiger Aufsummation der
einzelnen Atomstreuamplituden, wobei unterschiedliche
Atome an den Orten $\vec{r_j}$
andere Formfaktoren $f_j$ besitzen.
Somit folgt für die Streuamplitude oder auch Strukturamplitude genannt
\begin{align}
  A=\sum_j \symup{e}^{-2\pi i \vec{r}_j\left(\vec{k}-\vec{k}_0\right)} I_\mathrm{e}.
\end{align}
Durch Verwenden der Braggbedingung in der Form
\begin{align}
\vec{k}-\vec{k}_0=\vec{g}
\intertext{mit dem Reziprokengittervektor }
\vec{g}=
\end{align}


erweiterung auf einheiets zelle
-erweiterung auf gitter
-> Braggbedingung an Netzebenen
-> Strukturamplitude S





\subsection{Debye-Scherrer Methode zur Strukturbestimmung}
\label{subsec:Methoden}

\cite{sample}
