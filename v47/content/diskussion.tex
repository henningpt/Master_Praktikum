\section{Diskussion}
\label{sec:Diskussion}

Aus der Messung ergibt sich ein Wert für die Debyetemperatur von Kupfer
\begin{align}
  \theta_{D_{Kupfer}} = \SI{330(49)}{\kelvin}.
\end{align}
Die hohe Ungenauigkeit resultiert zum Einen daraus, dass für die Temperatur zur Berechnung von
$C_V$ ein $T_{mittel}$ verwendet wird und dabei ein Fehler relustiert, der proportional
zu den Temperaturschritt $\Delta T$ ist. Zum Anderen wurde
nur zu  Beginn der Messung der Heizstrom $I$ und die Heizspannung $U$
gemessen und die leichten autretenden Abweichungen während der Messung als Fehler betrachtet.
Um die Ungenauigkeit auf die Gemessene Debyetemperatur zu verringert muss somit die
Temperatur in geringeren Abständen gemessen werden und $I$ und $U$ zu jedem
Messwert erneut gemessen werden. (Wobei bei zu kleinen
Abständen die Thermische Fluktuationen die Messung ebenfalls verfäschen können)

Über die Gleichung \eqref{eqn:5} und \eqref{eqn:debye_temp}
ergibt sich ein Theoriewert von Kupfer der Debyetemperatur
\begin{align}
  \theta_{D_{theo}} = \SI{332.5}{\kelvin}.
\end{align}
Der Literaturwert für die Debyetemperatur von
Kupfer beträgt
\begin{align}
  \theta_{D_{lit}} = \SI{345}{\kelvin}   \text{\cite{debyetemp}.}
\end{align}
Die relative Abweichung des gemessenen Wertes beträgt zum Theoriewert somit
\begin{align}
a_{theo} = \SI{1(15)}{\percent}
\end{align}
und zum Literaturwert
\begin{align}
  a_{lit}= \SI{4(14)}{\percent}.
\end{align}
Der Theoriewert besitzt eine relative Abweichung von
\begin{align}
  a_{theo,lit}=\SI{3.6}{\percent}
\end{align}
zum Literaturwert.
Auf Grund der Abweichung der Messwerte zum Theoriewert
scheint die Messwerte die Theorie zu bestätigen bis auf die
hohe Ungenauigkeit des Messwertes. Jedoch zeigt
der Literaturwert eine gewisse Abweichung zu dem Theoriewert.
Dies deutet darauf hin, dass die Theorie noch nicht alle
effekte berücksichtig die und die Messung zu ungenau durchgeführt wurde.
Zusammenfassend lässt sich sagen, dass es bei genauerer Messung
duchaus möglich sein könnte, über den Versuchsaufbau die spezifische
Wärmekapazit einer Probe zu bestimmen.
