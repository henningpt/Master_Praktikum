\section{Diskussion}
\label{sec:Diskussion}

Aus der Messung ergibt sich ein Wert für die Debyetemperatur von Kupfer
\begin{align}
\theta_{D_{Kupfer}} = \SI{328(42)}{\kelvin}.
\end{align}
Die hohe Ungenauigkeit resultiert zum Einen daraus, dass für die Temperatur zur Berechnung von
$C_V$ ein $T_{mittel}$ verwendet wird und dabei ein Fehler resultiert, der proportional
zu den Temperaturschritten $\Delta T$ ist. Zum Anderen wurde
nur zu Beginn der Messung der Heizstrom $I$ und die Heizspannung $U$
gemessen und die auftretenden Abweichungen während der Messung als Fehler betrachtet.
Um die Ungenauigkeit auf die gemessene Debyetemperatur zu verringern muss somit die
Temperatur in geringeren Abständen gemessen werden und $I$ und $U$ zu jedem
Messwert erneut gemessen werden. Wobei jedoch zu kleine
Messabstände die Messung durch thermische Fluktuationen ebenfalls verfälschen können.
Ein weiteres Problem ist die thermische Isolation der Probe, um Wärmeverluste zu minimieren. Insbesondere das Kompensieren der Wärmestrahlung über
den Zylinder, stellt sich als Problematisch heraus, da Temperatur von Zylinder und Probe, um Kompensation zu gewährleisten, identisch sein müssen.
In der Tabelle \ref{tab:Messwerte} wird die Abweichung zwischen den Temperaturen aufgetragen, die bis zu $\SI{8.8}{\kelvin}$
beträgt. Um eine genauere Messung zu erhalten, sollte diese Abweichung verringert werden.

Über die Gleichung \eqref{eqn:5} und \eqref{eqn:debye_temp}
ergibt sich ein Theoriewert der Debyetemperatur für Kupfer von
\begin{align}
\theta_{D_{theo}} = \SI{332.5}{\kelvin}.
\end{align}
Der Literaturwert für die Debyetemperatur von
Kupfer beträgt
\begin{align}
\theta_{D_{lit}} = \SI{345}{\kelvin} \text{\cite{debyetemp}.}
\end{align}
Die relative Abweichung des gemessenen Wertes beträgt zum Theoriewert somit
\begin{align}
a_{theo} = \SI{1(13)}{\percent}
\end{align}
und zum Literaturwert
\begin{align}
a_{lit}= \SI{5(12)}{\percent}.
\end{align}
Der Theoriewert besitzt eine relative Abweichung von
\begin{align}
a_{theo,lit}=\SI{3.6}{\percent}
\end{align}
zum Literaturwert.
Aufgrund der Abweichung der Messwerte zum Theoriewert
scheinen die Messwerte die Theorie zu bestätigen. Allerdings
ist die hohe Ungenauigkeit zu beachten. Jedoch zeigt
der Literaturwert eine geringe Abweichung zu dem Theoriewert.
Dies deutet darauf hin, dass die Theorie möglicherweise noch nicht alle
Effekte berücksichtigt. Weiter weicht der gemessene Wert ebenfalls leicht von Literaturwert ab,
mögliche Ursachen dafür sind einmal die zuvor beschriebene Problematik der Kompensation von Wärmestrahlung
und weitere systematische Fehler.
Zusammenfassend lässt sich sagen, dass es bei genauerer Messung
durchaus möglich sein könnte, über den Versuchsaufbau
die Molwärme einer Probe zu bestimmen.
