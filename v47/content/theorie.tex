\section{Theorie}
\label{sec:Theorie}

\subsection{Zielsetzung}
\label{subsec:zielsetzung}
In dem hier beschriebenen Versuch wird untersucht,
inwiefern die Molwärme von der Temperatur abhängt.
Dazu werden zunächst drei Modelle zur theoretischen Beschreibung
der Molwärme von Festkörpern diskutiert und weiterhin experimentelle
Untersuchungen von Kupfer angestellt.
Anhand der daraus gewonnen Erkenntnisse, wird die Konstante $\Theta_{\text{D}}$,
die Debye-Temperatur, sowohl theoretisch, anhand des Debye-Modells,
als auch mit Hilfe der Messwerte bestimmt.

\subsection{Klassisches Modell}
\label{subsec:klassisch}
Bei einer rein klassischen Betrachtung, wird jedem Atom pro Freiheitsgrad eine mittlere
Energie von $\sfrac{k_{\text{B}} T}{2}$ zugeordnet. Die Atome im dreidimensionalen
Festkörper sind auf drei Bewegungsrichtungen, also sechs Freiheitsgrade, eingeschränkt, sodass
ein Mol eine mittlere Wärmeenergie von
\begin{align}
	U = 3 k_{\text{B}} T \, N_{\text{L}} = 3 R T\label{eqn:t2}
\end{align}
besitzt. Dabei bezeichnet $N_{\text{L}}$ die
Lohschmidtsche Zahl und
$R$ die allgemeine Gaskonstante.
Um die spezifische Molwärme $C_{\text{V}}$ zu erhalten,
wird die Energie eines Mols (siehe Gleichung \eqref{eqn:t2})
nach der Temperatur $T$ abgeleitet.
Es ergibt sich also
\begin{align}
	C_{\text{V}} = \frac{\partial U}{\partial T}
	\bigg\vert_{\text{V}} = 3 \, R \, . \label{eqn:t3}
\end{align}

Damit wird klar, dass in der klassischen Theorie,
die Molwärme sowohl unabhängig von der Beschaffenheit des
Festkörpers, als auch der Temperatur ist.


\subsection{Einsteinsches Modell}
\label{subsec:einstein}
Im einsteinschen Modell, wird, im Gegensatz zum klassischen
Modell, berücksichtigt, dass die Energie, mit
der die Atome im Festkörper Schwingen, quantisiert ist.
Ausgehend von einer festgelegten Kreisfrequenz $\omega$,
werden nur Energien der Form
\begin{align*}
	E_{n} = \hbar \omega n
\end{align*}
zugelassen. Wobei $n$ eine natürliche Zahl ist.
Diese möglichen Energiezustände sollen der Boltzmann-Statistik
folgen. Sodass der Energieerwartungswert für ein Atom durch
\begin{align}
	\langle u \rangle_{\text{Einstein}} = \frac{\sum^{\infty}_{n = 0}
	E_{n} e^{\frac{-E_{n}}{k_{\text{B}} T}}}{\sum^{\infty}_{n = 0}
	e^{\frac{-E_{n}}{k_{\text{B}} T}}} \label{eqn:5}
\end{align}

Damit ergibt sich dann für die Molwärme
\begin{align}
	3 R \frac{\hbar^{2} \omega^{2}}{k_{\text{B}}} \frac{1}{T^{2}}
	\frac{e^{\frac{\hbar \omega}{k_{\text{B}} T}}}{e^{\frac{\hbar \omega}{k_{\text{B}} T}} - 1}\label{eqn:t7}
\end{align}

Dabei gilt
\begin{align}
	\lim_{T \rightarrow \infty} C_{\text{V,Einstein}} = 3R \, . \label{eqn:t8}
\end{align}
Es lässt sich also erkennen, dass  für große Temperaturen
das einsteinsche Modell in das Klassische übergeht und für kleine Temperaturen
antiproportional zu $T^2$ verläuft.

\subsection{Debye-Modell}
\label{subsec:debye}
Zur besseren Beschreibung der Realität, wird im Debye-Modell,
nicht wie im Einsteinschen von einer konstanten Schwingungsfrequenz,
sondern mehreren ausgegangen.
Dazu wird eine Verteilungsfunktion $Z(\omega)$ eingeführt.
Diese beschreibt das Spektrum der Schwingungsfrequenzen.
Für die Molwärme gilt also
\begin{align}
	C_{\text{V}} = \frac{\mathrm{d}}{\mathrm{d}T}
	\int_{0}^{\omega_{\text{max}}} \mathrm{d}\omega Z(\omega)
	\frac{\hbar \omega}{e^{\frac{\hbar \omega}{k_{\text{B}} T} - 1}} \label{eqn:t9}
\end{align}
Im allgemeinen ist $Z(\omega)$  beliebig kompliziert.
Allerdings, wird im Debye-Modell die Annahme getroffen,
dass die Phasengeschwindigkeit unabhängig von Frequenz wie
Ausbreitungsrichtung ist.
Auf diese Weise vereinfacht sich $Z(\omega)$ derart, dass
ausschließlich die Eigenschwingungen eines Würfels(Kantenlänge $L$)
auf einem Intervall $\left[ \omega, \omega + \mathrm{d}\omega \right]$
zu berücksichtigen sind.
Damit ergibt sich
\begin{align}
	Z(\omega) \mathrm{d}\omega =  \omega^{2} \frac{L^{3}}{2 \pi^{2}}
	\left( \frac{1}{v_{\text{l}}^{3}} +
	\frac{2}{v_{\text{tr}}^{3}} \right) \mathrm{d}\omega \, . \label{eqn:t10}
\end{align}

Mit der longitudinalen Phasengeschwindigkeit $v_{\text{l}}$
und der transversalen $v_{\text{tr}}$.
In einem aus $N_{\text{L}}$ Atomen zusammengesetzten Festkörper,
existieren im dreidimensionalen Raum $3N_{\text{L}}$ Eigenschwingungen.
Aufgrund dieser Begrenzung, ist es notwendig, dass eine
Frequenz $\omega_{\text{D}}$, welche die obere Schranke der auftretenden
Frequenzen darstellt, existiert.
Diese wird als Debye-Frequenz bezeichnet und lässt sich
aus dem Zusammenhang
\begin{align}
	\int_{0}^{\omega_{\text{D}}} \mathrm{d}\omega = 3 N_{\text{L}} \label{eqn:t11}
\end{align}
gewinnen.
Es ergibt sich damit
\begin{align}
	\omega_{\text{D}}^{3} = \frac{18 \pi^{2} N_{\text{L}}}{L^{3}}
	\frac{1}{\frac{1}{v_{\text{l}}^{3}} + \frac{2}{v_{\text{tr}}}} \, .\label{eqn:t12}
\end{align}

Mithilfe von Gleichung \ref{eqn:t9}, \ref{eqn:t10} und \ref{eqn:t12}
lässt sich die spezifische Wärmekapazität als
\begin{align}
	C_{\text{V, Debye}} = \frac{\mathrm{d}}{\mathrm{d}T} \frac{9 N_{\text{L}} \hbar}{\omega_{\text{D}}^{3}}
	= \int_{0}^{\omega_{0}} \mathrm{d}\omega \frac{\omega^{3}}{e^{\frac{\hbar \omega}{k_{\text{B}} T}} - 1} \label{eqn:t14}
\end{align}
schreiben.
Mit den Definitionen
\begin{align}
	x &:= \frac{\hbar \omega}{k_{\text{B}} T} \label{eqn:t15} \\
	\frac{\theta_{\text{D}}}{T} &:= \frac{\hbar \omega_{\text{D}}}{\k_{\text{B}} T} \label{eqn:t16}
\end{align}
wird der Ausdruck für die Wärmekapazität \ref{eqn:t14} zu
\begin{align}
	C_{\text{V, Debye}} = 9 R \left(\frac{T}{\theta_{\text{D}}})\right)^{3}
	\int_{0}^{\frac{\theta_{\text{D}}}{T}} \frac{x^{4} e^{x}}{\left( e^{x} - 1 \right)^{2}} \, . \label{eqn:t17}
\end{align}

Bei einer Betrachtung von \ref{eqn:t17} im Grenzwert sehr großer Temperaturen,
ergibt sich für $C_{\text{V}}$ wieder die Konstante $3R$.

Da die Wärmekapazität kubisch mit der Temperatur ansteigt, kann der Beitrag freier Elektronen, der linear mit $T$ anwächst, im Festkörper vernachlässigt werden.



\cite{sample}
