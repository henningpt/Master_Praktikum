\section{Theorie}
\label{sec:Theorie}

\subsection{Zielsetzung}
\label{subsec:zielsetzung}
In dem hier beschriebenen Versuch wird untersucht,
inwiefern die Molwärme von der Temperatur abhängt.
Dazu werden zunächst drei Modelle zur theoretischen Beschreibung
der Molwärme von Festkörpern diskutiert und weiterhin experimentelle
Untersuchungen an  angestellt.
Anhand der daraus gewonnen Erkenntnisse Konstante $\Theta_{\text{D}}$,
die Debye-Temperatur, sowohl theoretisch, anhand des Debye-Modells,
als auch mit Hilfe der Messwerte bestimmt.

\subsection{Klassisches Modell}
\label{subsec:klassisch}
Bei einer rein klassischen Betrachtung, wird jedem Atom pro Freiheitsgrad eine mittlere
Energie von $\sfrac{k_{\text{B}i}}{2}$ zugeordnet. Die Atome im dreidimensionalen
Festkörper sind auf drei Bewegungsrichtungen, also sechs Freiheitsgrade, eingeschränkt, sodass
ein Atom eine mittlere Wärmeenergie von
\begin{align}
	<u> = 3 k_{\text{B}} T \lable{eqn:t1}
\end{align}
besitzt. Durch multiplizieren

\subsection{Einsteinsches Modell}
\label{subsec:einstein}


\subsection{Debye-Modell}
\label{subsec:debye}




\cite{sample}
