\newpage
\section{Diskussion}
\label{sec:Diskussion}

% T2
Die bestimmte Relaxationszeit $T_{2}$ wurde mit einer
relativen Abweichung von $\SI{2+-8}{\percent}$
zum Literaturwert bestimmt.
Wobei jedoch einfachheitshalber nicht nur, wie in Kapitel \ref{subsubsec:T2mess}
gefordert, nach jedem geraden
Puls die Echoamplitude verwendet wird, sondern Maxima von bestimmten Invervallen verwendet werden.
Trotzdem ist es möglich so die Relaxationszeit $T_{2}$ über die
Meibomm-Gill-Methode zu bestimmen.
% Die auftretende Abweichung lässt sich vor allem dadurch begründen,
%  dass, wie in Abbildung \ref{fig:t2_gesamt}
% zu erkennen, sicherlich nicht immer die richtigen Messpunkte
% als Echos identifiziert wurden.
% Diffusion

Wie erwartet, ergibt sich der Literaturwert für die Diffusionskonstante $D$
mit einer Abweichung von $\SI{0.00(3)}{\percent}$, da die Halbwertszeit $t_{\frac{1}{2}}$
genau auf den Literaturwert angepasst wurde.
Für die gegebene Apparatur liegt die Halbwertszeit in einem Intervall
von $\SI{70}{\micro\second}-\SI{250}{\micro\second}$ \cite{talk}.
Da die ermittelte Halbwertszeit von $\SI{355(6)}{\micro\second}$
außerhalb des Invervalles liegt, kann daraus gedeutet werden,
dass die Diffusionskonstante des Literaturwert für die Apparatur nicht genau zutrifft.
Mögliche Ursachen dafür sind die Temperaturabhängikeit und unterschiedliche
Magnetfeldgradienten bei der Messung.

Der Moleküladius über den Literaturwert der Diffusionskonstante $D$
liegt im Bereich zwischen den beiden anderen Methoden. Deshalb eignet sich die
Difussionsmeessung ebenfalls, um eine Abschätzung des Moleküladius vorzunehmen.
Die relativen Abweichungen $f$ betragen jeweils
\begin{align*}
  f_{\text{hexagonal}}&=\SI{14(3)}{\percent}\\
  f_{\text{van-der-Waal}}&=\SI{4(3)}{\percent}.
\end{align*}
Deutlich wird, dass die Abweichung zur hexagonal dichtesten Packung
höher ist, als die zum van-der-Waal-Gases.
Dies lässt sich dadurch erklären, dass bei dem van-der-Waals-Gas
ebenfalls
wie bei der Diffusionsmessung eine Bewegung der Teilchen vorliegt
und nicht star angeordnete Teilchen wie bei der dichtesten Packung.
% Dieser Unterschied lässt sich durch die zuvor falsch bestimmte Diffusionskonstante zurückführen.



% Bei der Messung der Diffusionskonstanten $D$ wurde ein um $4$ Zehnerpotenzen vom Literaturwert variierender Wert ermittelt.
% Diese lässt auf einen unbekannten systematischen Fehler schließen, da der Graph der Regression eine so starke Abweichung nicht erklären kann.
% T1
Bei der Messung der Relaxationszeit $T_{1}$ wurde ein Wert mit einer Abweichung von $\SI{25+-4}{\percent}$ zum Literaturwert bestimmt.
Eine Fehlerquelle, die diese Abweichung erklärt,
ist, dass sowohl Temperatur als auch Larmorfrequenz bei dem Literaturwert unbekannt sind.
Desweiteren können Feldinhomogenitäten diesen Wert ebenfalls beeinflussen.
