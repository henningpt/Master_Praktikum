\newpage
\section{Diskussion}
\label{sec:Diskussion}

% T2
Die bestimmte Relaxationszeit $T_{2}$ wurde mit einer relativen Abweichung von $\SI{32+-8}{\percent}$ zum Literaturwert bestimmt.
Die auftretende Abweichung lässt sich vor allem dadurch begründen, dass, wie in Abbildung \ref{fig:t2_gesamt}
zu erkennen, sicherlich nicht immer die richtigen Messpunkte als Echos identifiziert wurden.
% Diffusion
Bei der Messung der Diffusionskonstanten $D$ wurde ein um $4$ Zehnerpotenzen vom Literaturwert variierender Wert ermittelt.
Diese lässt auf einen unbekannten systematischen Fehler schließen, da der Graph der Regression eine so starke Abweichung nicht erklären kann.
Der Moleküladius wurde im Femtometerbereich ermittelt, allerdings liegt der Vergleichswert im realistischeren Angstrom Bereich.
Dieser Unterschied lässt sich durch die zuvor falsch bestimmte Diffusionskonstante zurückführen.
% T1
Bei der Messung der Relaxationszeit $T_{1}$ wurde ein Wert mit einer Abweichung von $\SI{27+-4}{\percent}$ zum Literaturwert bestimmt.
Eine Fehlerquelle die diese Abweichung erklärt,
ist dass das verwendete Magnetfeld möglicherweise nicht konstant oder nicht ausreichend stark war.
