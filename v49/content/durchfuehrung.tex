\newpage
\section{Durchführung}
\label{sec:Durchführung}
\subsection{Aufbau und Justage}
Für die folgenden beschriebenen
Messungen wird die in Abbildung \ref{fig:teachspin}
dargestellte Teachspin Apparatur verwendet.
\FloatBarrier
\begin{figure}
  \includegraphics[width=0.7\textwidth]{teachspin.PNG}
  \caption{The PS2-A Spectrometer: Magnet, Mainframe and PS2 Controller}
  \label{fig:teachspin}
\end{figure}
\FloatBarrier
Zur Justage der Pulslänge wird zunächst
in die Apparatur eine Wasserprobe
mit paramagnetischen Zentren eingefügt, um
die Relaxationszeiten und somit die Wiederholungszeit $P$ der Messung zu verkürzen.
Dann werden Startparameter
für die Gradientenspule,
den A-Puls($\SI{90}{\degree}$-Puls)
und der Frequenz des HF-Feldes gewählt
und versucht die Resonanzfrequenz der Probe
zu treffen, sodass der FID auf dem Oszilloskop beobachtet werden kann. Des Weitern werden die
Parameter wie Pulslänge, Phase und Shim-Parameter so variiert, dass
der FID optimiert wird.
\subsection{Durchführung}
Gelingt dies wird der B-Puls($\SI{180}{\degree}$-Puls),
der die doppelt Pulslänge besitzt wie der A-Puls,
programmiert und es kann ein Echo-Signal beobachtet werden.
Danach wird die Probe durch bidestiliertes Wasser
und die Wiederholungszeit aufgrund der größeren Relaxationszeit
erhöht. Es wird jeweils eine Pulssequenz der Carr-Purcell-
und Meiboom-Gil-Methode bei einem festem $\tau$ aufgenommen.
Danach wird eine Messung für die Bestimmung der Relaxationszeit $T_1$
durchgeführt. Zuvor werden die Pulslängen so getauscht, dass , wie in Kapitel \ref{subsec:t1}
beschrieben, der A-Puls zu einem $\SI{180}{\degree}$-Puls
und der B-Puls zu einem $\SI{90}{\degree}$-Puls wird.
Dann wird für verschiedene Pulsabstände $\tau$ die Echoamplitude aufgenommen.
Für die Diffusionskoeffizienten Messung
werden die Pulse wieder zurückgetauscht
und es wird der maximale Magnetfeldgradient
in z-Richtung eingestellt. Es wird die Halbwertsbreite $t_{\sfrac{1}{2}}$
des Spin Echos gemessen und mit Hilfe
des Spin-Echo-Verfahrens Messwerte zur Bestimmung
des Diffusionskoeffizienten $D$ aufgenommen.
Zusätzlich zu den anderen Messungen
wird noch die Viskosität $\rho$
der Probe mit Hilfe eines Viskosimeters
% ???????? noch beschreibung des Messvorgangs notwendig?
bei gleicher Temperatur wie bei der
Diffusionskoeffizienten-Bestimmung gemessen.
