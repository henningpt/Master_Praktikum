\FloatBarrier
\section{Diskussion}
\label{sec:Diskussion}

\paragraph{Erzeugen und Untersuchung einer amplitudenmodulierten Schwingung mit
Hilfe eines Ringmodulators}
Es ist möglich mit einem Ringmodulator
eine amplitudenmodulierte Schwingung
mit Trägerunterdrückung zu erzeugen.
Dies ist einerseits daran zu erkennen,
dass die Amplitude des generierten Signals abhängig
von der Frequenz des Modulationssignals schwankt.
%
% \paragraph{Untersuchung des Frequenzspektrums einer
% amplitudenmodulierten Schwingung}
Anderseits ist im Frequenzspektrum zu erkennen, dass sich zwei
Peaks mit den geforderten Frequenzen $f_{\text{T}}\pm f_{\text{M}}$
bilden
und kein Peak an der Stelle der Trägerfrequenz
auftritt.
\paragraph{Erzeugen einer amplitudenmodulierten Schwingung
mit Hilfe einer Gleichrichterdiode}
Mit Hilfe einer Gleichrichterdiode kann
die erfolgreiche Erzeugung eines Signals mit Trägerabstrahlung
realisiert werden.
Dies ist einerseits durch das typische Bild im Zeitraum eines
Amplitudenmodulierten Signals, dessen negative Halbwellen
abgeschnitten wurden
und andererseits durch auftretenden Frequenzen höherer
Ordnung(siehe Kapitel \ref{subsubsec:modulationsschaltungen})
im Frequenzraum, sowie der zu beobachtenden
Trägerabstrahlung, zu erkennen.

Desweiteren kann der Modulationsgrad $m$ des Signals
über zwei verschiedene
Methoden bestimmt werden. Dabei ergeben sich mit
\begin{align*}
m=\num{0.64(9)}(\textbf{Amplitude})& &\text{und}& & m = \num{0.182(1)}\notag
(\textbf{Leistung})
\end{align*}
zwei unterschiedliche Werte. Dies lässt sich dadurch erklären das bei dem Oszillographen
nicht alle Oberwellen des Signals herausgefiltert werden konnten.
Somit lässt sich davon ausgehen, dass der Wert, der über das Frequenzspektrum bestimmt
wurde, zuverlässiger ist.
\paragraph{Erzeugen einer frequenzmodulierten Schwingung}
Der Ringmodulator mit Laufzeitkabel eignet sich zur Erzeugung
eines frequenzmodulierten Signales, wie anhand des aufgenommenen Bilds
am Oszillographen zu erkennen ist.
Es kann sowohl über
die Verschmierung als auch über eine Frequenzanalyse
ein Modulationsgrad bestimmt werden.
Es ergeben sich die beiden Werte
\begin{align}
  m = \num{0.039(4)}(\textbf{Verschmierung})& &\text{und}& &m=\num{0.0452(1)}(\textbf{Leistung}).
\end{align}
Aufällig ist die deutlich geringere relative Abweichung der beiden Werte,
als bei der Bestimmung für die Amplitudenmodulation.
Dies lässt sich zum Teil dadurch begründen, dass beim Betrachten der Verschmierung nicht die
Amplituden und somit auch nicht die unerwünschten Oberschwingungen eingehen,
wie es bei der Amplitudenmodulation der Fall ist.
% Dies ist möglicherweise dadurch zu erklären, dass
% die bei der Bestimmung durch Frequenzanalyse, Näherungen vorgenommen
% wurden, deren Gültigkeit infrage gestellt werden könnte, da
% $m \, \sfrac{\omega_{\text{T}}}{\omega_{\text{T}}}\ll1$
% nicht eindeutig erfüllt ist.

\paragraph{Untersuchung der Phasenabhängigkeit eines
phasenempfindlichen Gleichrichters}
Bei der Untersuchung der Phasenabhängigkeit, des am phasenempfindlichen
Gleichrichters gemessenen Gleichstroms kann der Zusammenhang $U \propto \cos \phi$ anhand der
aufgenommen Messwerte bestätigt werden.

\paragraph{Demodulation einer amplitudenmodulierten Schwingung
mit Hilfe eines Ringmodulators}
% 1 nices verfahren.
Mithilfe des Ringmodlators kann erfolgreich
ein zuvor amplitudenmoduliertes Signal ohne signifikate Amplitudenverluste
demodliert werden.
Bei gegebenem Trägersignal eignet sich somit dieses Verfahren für die
Demodualtion eines Signals.

\paragraph{Demodulation einer amplitudenmodulierten Schwingung
mit Hilfe einer Gleichrichterdiode}
% nicht ganz so nice
Auch mit Hilfe einer Gleichrichterdiode in Kombination mit einem
Tiefpass, kann das Signal demoduliert werden.
Allerdings nur in Verbindung mit nennenswerten
Amplitudenverlust und Frequenzverdoppelung.
Der Amplitudenverlust kann dadurch erklärt werden, dass sowohl
an Diode als auch Tiefpass Spannung abfällt.

Die verdoppelte Frequenz resultiert aus dem Abschneiden
der Halbwellen der Schwebung durch die Diode.
Da die Schwebung die doppelte Frequenz besitzt, ergibt sich nach der Filterung
durch den Tiefpass ein Signal mit ebenfalls der doppelten Frequenz.
\paragraph{Demodulation einer frequenzmodulierten Schwingung
mit Hilfe eines Flankendemodulators}
% eher so meh
Die Umwandlung eines frequenzmodulierten in ein
amplitudenmoduliertes Signal ist durch einen Flankendemodulator
und anschließender Demodulation mit Gleichrichter
möglich. Jedoch verringern die
vielen Umwandlungsprozesse die Amplitude des Signals.
% Modulation ist super toll!(kleines hündchen)
