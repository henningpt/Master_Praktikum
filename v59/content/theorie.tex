\section{Theorie}
\label{sec:Theorie}
Die Modulation und Demodulation
elektrischer Schwingungen wird
zur Informationsübertragung genutzt.
Dabei wird mit Hilfe der Modulation
(siehe Kapitel \ref{subsubsec:Modulationsverfahren})
Amplitude, Frequenz oder auch die Phase
der Welle dem Nachrichtensignal entsprechend
verändert.
Um der modulierten Welle die Information
wieder zu entnehmen werden am Empfangsort
Demodulationsverfahren verwendent.

\subsection{Modulationsverfahren}
\label{subsec:Modulationsverfahren}
Es existeren in der Hochfrequenztechnik einige
Verfahren zur Modulation, die Unterschiedlichen
Anforderungen erfüllen. Grob können diese
jedoch in 2 Verfahren unterteilt
werden, einmal die Amplitudenmodulation \ref{subsubsec:Amplitudenmodulation}
und die Frequenzmodulation \ref{subsubsec:Frequenzmodulation}.


\subsubsection{Amplitudenmodulation}
\label{subsubsec:Amplitudenmodulation}
Die Amplitudenmodulation kennzeichnet sich
dadurch, dass sich dabei die Signalamplitude
des Trägers $U_{\text{T}}(t)$ periodisch verändert.
Die Veränderung der Amplitude ist dabei gegeben
durch die Frequenz der Modulationsschwingung
$U_{\text{M}}(t)$.
Ausgehend von den Darstellungen der Signale
\begin{align}
  U_{\text{T}} &= \hat{U}_{\text{T}} \, \cos \omega_{\text{T}} \, t \\
  U_{\text{M}} &= \hat{U}_{\text{M}} \, \cos \omega_{\text{M}} \, t
\end{align}
mit den jeweiligen Kreisfrequenzen $\omega_{\text{T}}$ und $\omega_{\text{M}}$,
ergibt sich für das aus der Amplitudenmodulation resultierende Signal
\begin{align}
  \label{eqn:1}
  U_{3}(t) = \hat{U}_{\text{T}} \, \left( 1 + m \, \cos \omega_{\text{M}} t \right) \, .
\end{align}

Dabei ist $m$, der zu $U_{\text{M}}Q$ proportionale, Modulationsgrad.
Der zugehörige Proportionalitätsfaktor heißt $\gamma$.
Es gilt $0 \leq m \geq 1$.
In Abbildung \ref{fig:amplitudenmodulation_1} ist das Signal aus Gleichung
\eqref{eqn:1} dargestellt.
Dieses lässt sich so Umformen, dass die auftretenden Frequenzen direkt
ablesbar sind
\begin{align}
  \label{eqn:2}
  U_{\text{T}} &= \hat{U}_{\text{T}} \, \left( \cos \omega_{\text{T}} t \, + \, \frac{m}{2} \cos t\left( \omega_{\text{T}} + \omega_{\text{T}} \right) \, + \, \frac{m}{2}  m  \cos t \left( \omega_{\text{M}} - \omega_{\text{T}} \right)\right) \, .
\end{align}
Im Gegensatz zu den beiden Frequenzen $\omega_{\text{T}} + \omega_{\text{M}}$ und \omega_{\text{T}} - \omega_{\text{M}}, tritt die Frequenz $\omega_{\text{T}}$
unabhängig von $U_{\text{M}}$ auf und spielt somit
für die Informationsübertragung keine Rolle. Daher soll $\omega_{\text{T}}$
aus Gründen der Energieeffizienz herausgefiltert werden.
Aus den beiden übrigen Frequenzenlinien werden, wenn das Modulationssignal
aus mehreren Frequenzen zusammengesetzt ist, Frequenzbänder.
Diese tragen jeweils die selbe Information, sodass eines der beiden ohne
Informationsverlust unterdrückt werden kann(\textit{Einseitenbandmodulation}).\\ \\

Probleme der Amplitudenmodulation stellen die geringe Toleranz gegenüber
Störungen, sowie die geringe Verzerrungsfreiheit.


\subsubsection{Frequenzmodulation}
\label{subsubsec:Frequenzmodulation}
Bei der Frequenzmodulation wird die
momentane Schwingungsfreuquenz
abhängig
von dem Modulationssignal und die
Amplitude der Schwingung bleibt konstant.
Die Schwingung nimmt dann folgende Form an
\begin{align}
U(t) = \hat{U} \sin\left(\omega_{\text{T}} t + m\frac{\omega_{\text{T}}}{\omega_{\text{M}}} \cos\omega_{\text{M}}t \right),\label{eqn:3}
\intertext{mit der Momentanfrequenz}
f(t)=\frac{\omega_{\text{T}}}{2\pi} \left(1-m\sin\omega_{\text{M}}t\right)\label{eqn:f}
\end{align}
wobei $m$ dem Modulationsgrad
entspricht.
Die Variationsbreite
der Schwingungsfrequenz
wird Frequenzhub genannt und entspricht
$m\sfrac{\omega_{\text{T}}}{2\pi}$.
Die Abbildung \ref{fig:Frequenzmodulation}
enthält einen beispielhaften Verlauf
einer Frequenzmodulation.

% \begin{figure}
%   \centering
%   \includegraphics[width=0.7\textwidth]{figure/f_moduliert.PNG}
%   \caption{Zeitlicher Verlauf einer frequenzmodulierten Schwingung.}
%   \label{fig:Frequenzmodulation}
% \end{figure}

Wird nur der Fall eines
niedrigens Frequenzhubes
\begin{align*}
  m\frac{\omega_{\text{T}}}{\omega_{\text{M}}} \ll 1
\end{align*}
betrachtet, kann die Gleichung
\eqref{eqn:3} durch Nährungen
in die Form
\begin{align}
\hat{U}(t)=\hat{U}\left(\sin \omega t + \frac12 m\frac{\omega_{\text{T}}}{\omega_{\text{M}}}\cos t(\omega_{\text{T}}+\omega_{\text{M}})
+ \frac12 m\frac{\omega_{\text{T}}}{\omega_{\text{M}}}\cos t(\omega_{\text{T}}-\omega_{\text{M}}) \right)
\end{align}
gebracht werden.
Daraus kann entnommen werden, dass
bei schwach fequenzmodulierten Schwingungen
ebenso wie bei amplitudenmodulierten
drei Teilschwingungnen mit den Frequenzen $\omega_{\text{T}},\omega_{\text{T}}+\omega_{\text{M}}$
und $\omega_{\text{T}}-\omega_{\text{M}}$ existieren.
Dabei sind die Seitenlinien $\omega_{\text{T}}\pm\omega_{\text{M}}$
um $\sfrac{\pi}{2}$ zur Trägerschwingung verschoben.

\subsection{Schaltungen}
\label{subsec:Schaltungen}
Um die im Kapitel \ref{subsec:Modulationsverfahren}
beschriebenen Modulationsverfahren so wie Demodulationsverfahren zu realisiert, werden
bestimmte elektronische Schaltungen
benötigt auf die im folgenden näher eingegangen werden soll.
\subsection{Modulationsschaltungen}
\label{subsubsec:Modulationsschaltungen}
\paragraph{Amplitudenmodulation}
Für die Amplitudenmodulation
wird ein Gerät benötigt, das ein
Produkt zweier Spannungen bildet.
Bauteile mit einer nicht-lineare Kennlinie
erfüllen diese Bedingung wie zum Beispiel ein
Diode.
Somit lässt sich ein Modulatorschaltung
wie in Abbildung \ref{fig:diode}
mit einer Diode realisieren.
\begin{figure}
  \centering
  \includegraphics[width=0.5\textwidth]{figures/diode.PNG}
  \caption{Primitive Modulatorschaltung mit Diode.}
  \label{fig:diode}
\end{figure}
Über die Reihenentwicklung der
Diodenkennlinie ergibt der gewünschte
Term $U_{\text{T}}U_{\text{M}}$ jedoch
treten dabei weitere störende Terme
auf wie z.B. $U_{\text{M}}, U_{\text{M}}^2$ und $U_{\text{T}}^2$.
Da diese normalerweise Frequenzen
besitzen, die weit außerhalb des
Freuenzbandes $[\omega_{\text{T}}-\omega_{\text{M}},\omega_{\text{T}}+\omega_{\text{M}}]$ liegen,
können diese mit Hilfe eines Bandfilters unterdrückt werden.
Eine bessere Methode zur  Amplitudenmodulation
stellt ein
Ringmodulator dar, da dieser
erst garnicht die unerwünschten komponenten
erzeugt.
Ein Ringmodulator besteht aus
4 zu einem Ring geschaltenten Dioden,
wie in Abbildung
\ref{fig:Ringmodulator} dargestellt.


\begin{figure}
  \centering
  \includegraphics{figures/Ringmodulator.PNG}
  \caption{Schaltbild für ein Ringmodulator.}
  \label{fig:Ringmodulator}
\end{figure}

Für übereinstimmende elektrische Eigneschaften
der 4 Dioden und einer Modulationsspannung $U_{\text{M}}=0$
tritt zwischen den Punkten $\alpha$ und $\beta$






\paragraph{Frequenzmodulation}
Für eine Frequenzmodulation, die einen geringen Frequenzhub besitzt,
kann die in Abbildung \ref{fig:Ringmodulator_frequenz} dargestellte
Schaltung verwendet werden.
Der Ringmodulator erzeugt die Frequenzen $\text{\faxmachine}$


\subsubsection{Demodulationsschaltungen}
\label{subsubsec:demodulationschaltungen}
