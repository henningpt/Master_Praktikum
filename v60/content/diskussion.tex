\newpage
\section{Discussion}
\label{sec:Diskussion}
A general difficulty is the precise adjustment of the diode laser's parameters
for example grating orientation and the diode current.
As described in \ref{sec:LED_Laser}
the transition from LED to a laser diode
can be successfully observed at a threshold current
of $\SI{33,7}{\milli\ampere}$.
The precision of the threshold current significant
dependents on experimenter subjective perception of
the laser's brightness
and the ability to fine tune the laser with the different knobs.
Furthermore, the rubidium fluorescence can be observed
as expected (see \ref{fig:fluores}).
Due to the unfortunate
data loss, there is no evidence of the successful observation of the
full rubidium absorption spectrum with the first method.
On the contrary the result of the second method is
recorded properly and shows a full trace over the rubidium absorption spectrum.
This shows that a diode laser can be
adjusted to sweep over a frequency band without mode hops by changing the current
and the length of the external cavity. Hence, it is possible to observe an
absorption spectrum over an extended range.
