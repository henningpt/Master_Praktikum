\section{Implementation of the Experiment}
\label{sec:Durchführung}
To observe and record the Rubidum absoptionspectrum at the ende
of the experiment
serveral premeasurements with different setups are necessary.
% Used devices are two Photodiodes and
% of cause a diode laser
A dark room for the experiment is recommended to minize the
interfering light sources.
The used diode laser has two knobs to controll the alignment of grating
as showen in figure \ref{fig:knobs}.
For the horizontal alignment the side knob is turned. The
top knob changes the vertical alignment.
\begin{figure}
  \centering
  \includegraphics[width=0.7\textwidth]{Laserknobs.png}
  \caption{It is shown top knob and side knob of the diode laser to align the grating.\cite{V61}}
  \label{fig:knobs}
\end{figure}


\subsection{Setup}
\label{subsec:setup}
The fist step is to determine the threshold current
of the diode laser.
Therefore a index card is placed in laser beam and the CCD Camera
is focused on the point where the card intercpets the beam.
The setup is displayed
in figure \ref{fig:setup1}.

\begin{figure}
  \centering
  \includegraphics[width=0.7\textwidth]{setup1.png}
  \caption{Schematical setup to observe the threshold current of the diode laser.\cite{V61}}
  \label{fig:setup1}
\end{figure}
By increasing the current, starting at zero, the transition from
a normal LED to a laser diode be observed.
The threshold current is at the
transition point where the light spot becomes significant brighter.
Futhermore by adjusting the top knob the threshold current can be lowered.

% The thresholf current can verrigert werden durch
%( irgendwas mit den Rädchen)

After the lowest possible threshold current is determined the
index card is removed and
the Rubidum absorption cell
is placed in the laserbeam instead.
A schematical setup is shown in figure \ref{fig:setup2}
Additionally the diode lasers ramp generator and  piezo controller is wired as shown in figure \ref{fig:dl_controll}.
The camera is now
focus on the absorption cell and
the Diode is operated above the threshold current.
\begin{figure}
  \centering
  \includegraphics[width=0.7\textwidth]{setup2.png}
  \caption{Schematical setup to observe the Rubidium florescence.\cite{V61}}
  \label{fig:setup2}
\end{figure}

\begin{figure}
  \centering
  \includegraphics[width=0.7\textwidth]{wiring.png}
  \caption{Wiring of the piezo controllelement to observe the Rubidium florescence.\cite{V61}}
  \label{fig:dl_controll}
\end{figure}

By adjusting the side knob and the current
the fluorescence of Rb atoms in the absoption cell
can be observed
as a fluorescence flashing along the laserbeam.
Besides the observation of the camera a photodiode
is installed behind the absoption cell
measuring the intensity
of the out coming laserbeam.
 and
oscilloscope


Simultaneous current and piezo modulation

\begin{figure}
  \centering
  \includegraphics[width=0.7\textwidth]{setup3.png}
  \caption{Schematical setup to observe the Rubidium absoptionspectrum.\cite{V61}}
  \label{fig:setup3}
\end{figure}


\begin{figure}
  \centering
  \includegraphics[width=0.7\textwidth]{wiring2.png}
  \caption{Wiring of the controllelement to observe the Rubidium florescence.\cite{V61}}
  \label{fig:dl_controll2}
\end{figure}


\subsection{Procedure}
\label{subsec:procedure}
