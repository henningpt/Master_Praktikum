\section{Auswertung}
\label{sec:Auswertung}



\subsection{Stabilitätsbedingung}
\label{subsec:Stabil}



\subsection{TEM-Moden}
\label{subsec:tem}
Die longitudialen Moden

\begin{figure}
  \includegraphics{/path/to/figure}
  \caption{}
  \label{}
\end{figure}


Weiterhin werden zwei transversale Moden des Lasers untersucht.
Zum einen die \textbf{$\text{TEM}_{00}$},
deren Intensiätsmaximum in der Mitte des Strahls lokalisiert ist(siehe Abbildung \ref{fig:mode00}).
An die aufgenommenen Messwerte wird ein Gaußfit gemäß Gleichung \ref{eqn:TEM00} vorgenommen.
In Abbildung \ref{fig:mode00} ist der entsprechende Graph neben den Messwerten dargstellt.
Es ergeben für die Fitparameter
\begin{align}
  I_0=\SI{}{\nano\ampere}\\
  d_0=\SI{}{\milli\meter}
\end{align}
\begin{figure}
  \centering
  \includegraphics{build/mode00.pdf}
  \caption{Gemessene Intensitätsverteilung der \textbf{$\text{TEM}_{00}$}-Mode und Gaußfit an die Messwerte.}
  \label{fig:mode00}
\end{figure}

Zum anderen wird die \textbf{$\text{TEM}_{01}$} untersucht, deren Intensitätsmaxima durch
zwei gleich weit vom Strahlzentrum entfernte Gaußpeaks gegeben sind.
An die aufgenommenen Messwerte wird ein Gaußfit gemäß Gleichung \ref{eqn:TEM01} vorgenommen.
In Abbildung \ref{fig:mode01} ist der entsprechende Graph neben den Messwerten dargstellt.
\begin{align}
  I_{(0,1)}(d)=I_{0,1}\exp\left(-2\left(\frac{d-d_{0,1}}{\omega_1}\right)^2\right)+I_{0,2}\exp\left(-2\left(\frac{d-d_{0,2}}{\omega2}\right)^2\right) \label{eqn:TEM01asy}
\end{align}
Es werden die Fitparameter
 \begin{align}
   I_{0,1}&=\SI{}{\nano\ampere}&   d_{0,1}&=\SI{}{\milli\meter}\\
   I_{0,2}&=\SI{}{\nano\ampere}&   d_{0,2}&=\SI{}{\milli\meter}
 \end{align}
gefunden.
\begin{figure}
  \centering
  \includegraphics{build/mode01.pdf}
  \caption{Gemessene Intensitätsverteilung der \textbf{$\text{TEM}_{01}$}-Mode und Gaußfit an die Messwerte.}
  \label{fig:plot}
\end{figure}



\subsection{Polarisation}
\label{subsec:Polarisation}
Die Messwert zur Bestimmung der Polarisation des Lasers sind in Tabelle \ref{tab:polarisation} aufgelistet.
In Abbildung \ref{fig:polarisation} wird die Intensität $I$ in Ahängigkeit von dem Polarisationswinkel $\phi$
aufgetragen und die Funktion
\begin{align}
I(\phi)=I_0 \cos\left(\phi-\phi_0\right)
\end{align}
an die Messwerte an gefittet.
Für die Funktion ergeben sich die Parameter
\begin{align}
I_0  & = \SI{142.4(6)}{\nano\ampere},\\
\phi & = \SI{68.3(2)}{\degree}.
\end{align}

\begin{table}
  \centering
  \caption{Messwerte der Intensität in Abhängigkeit von dem Winkel des Polarisationsfilters.}
  \label{tab:polarisation}
  \begin{tabular}{c c | c c}
\toprule
    $\phi / \si{\degree}$ & $I/\si{\nano\ampere}$ & $\phi / \si{\degree}$ & $I/\si{\nano\ampere}$\\
\midrule
0	&	24.8   & 190	&	29.2 \\
10	&	51.2 &  200	&	38.0  \\
20	&	83.3 &  210	&	64.5  \\
30	&	119.8&  220	&	71.0   \\
40	&	144.9&  230	&	97.9   \\
50	&	144.1&  240	&	130.1  \\
60	&	111.7&  250	&	112.2  \\
70	&	201.3&  260	&	125.6  \\
80	&	209.4& 270	&	96.9    \\
90	&	130.1& 280	&	76.0    \\
100	&	101.7& 290	&	88.6    \\
110	&	74.8 & 300	&	51.0   \\
120	&	45.3 & 310	&	38.5   \\
130	&	26.4 & 320	&	18.4   \\
140	&	12.4 & 330	&	4.4    \\
150	&	2.8  & 340	&	0.5   \\
160	&	0.5  & 350	&	4.8   \\
170	&	2.5  & 360	&	26.1  \\
180	&	11.7 &  &  \\
\bottomrule
\end{tabular}
\end{table}


\begin{figure}
  \centering
  \includegraphics{build/polarisation.pdf}
  \caption{Intensität in Abhängigkeit des Polarisationswinkel.}
  \label{fig:polarisation}
\end{figure}




\subsection{Wellenlängenbestimmung}
\label{subsec:wellenlaenge}

Die Wellenlänge der charakteristischen roten Linie des He-Ne-Lasers, wird mit Hilfe
von zwei optischen Gittern ($g_{1} = \SI{80}{\per\centi\meter} \ g_{2} = \SI{100}{\per\centi\meter}$) bestimmt.
Es wird jeweils ein Wert zu jedem Maximum bestimmt. Aus dem Mittel der Werte aus den einzelnen Maxima
wird für jedes Gitter eine Wellenlänge berechnet.
In Tabelle \ref{tab:wellenlaenge} sind jeweils die Werte für Schirmabstand $L$ und Abstände der Maxima
zum Maximum $0.$-Ordnung $d_{\text{i}}$ zusammen mit den bestimmten Wellenlängen angegeben.\\
Es ergibt sich eine mittlere Wellenlänge von
\begin{align*}
  \lambda_{1} &= \SI{6.53(2) e-7}{\nano\meter}\\
  \lambda_{2} &= \SI{6.58(1) e-7}{\nano\meter} \, .
\end{align*}


\begin{table}
\centering
  \caption{Mit zwei optischen Gittern bestimmte Wellenlängen mit jeweiligen relevanten Größen.}
  \label{tab:wellenlaenge}
\begin{tabular}{c|c c || c c}
  & \multicolumn{2}{c}{$g_{1} = \SI{80}{\per\centi\meter}$} & \multicolumn{2}{c}{$g_{2} = \SI{100}{\per\centi\meter}$} \\
\toprule
Ordnung $i$ &    $d_i / \si{\centi\meter}$  & $\lambda / \si{\nano\meter}$ &    $d_i / \si{\centi\meter}$  & $\lambda / \si{\nano\meter}$ \\
\midrule
1 & \phantom{0}7.2 \pm	\, 0.1	&	637.4	\pm	\,8.8	&	\phantom{0}9.2  	\pm	\, 0.1	&	650.0	\pm	\,7.0   \\
2 & 14.5	         \pm	\, 0.1	&	634.9	\pm	\,4.3	&	18.5	            \pm	\, 0.1	&	645.9	\pm	\,3.4   \\
3 & 21.8	         \pm	\, 0.1	&	630.8	\pm	\,2.8	&	27.8	            \pm	\, 0.1	&	639.3	\pm	\,2.2   \\
4 & 29.2	         \pm	\, 0.1	&	630.5	\pm	\,2.1	&	43.0	            \pm	\, 0.1	&	724.6	\pm	\,1.5   \\
5 & 41.2	         \pm	\, 0.1	&	697.4	\pm	\,1.6	&	48.2	            \pm	\, 0.1	&	643.4	\pm	\,1.2   \\
6 & 49.5	         \pm	\, 0.1	&	685.8	\pm	\,1.2	&	59.5	            \pm	\, 0.1	&	644.1	\pm	\,0.9   \\
\bottomrule
\end{tabular}
\end{table}
