\section{Diskussion}
\label{sec:Diskussion}

% Stabilitätsbedingung
Bei der Untersuchung der Stabilitätsbedingung wurden für $g_{1} \, g_{2}$
Werte zwischen $0.39$ und $0.05$ erreicht.
Wie in Abbildung \ref{fig:oberaffengeilerplot} gut zu erkennen, liegen damit alle Werte
in einem für Stabilität charakteristischen Bereich.
Weiterhin lässt grob erkennen, dass die gemessenen Leistungswerte($\propto$Intensität)
im Bereich Bereich um $0.5$ deutlich größer sind als am Rand des stabilen Bereichs nahe $0$.
Somit stimmen die Messungen damit überein, dass $g_{1} \, g_{2}$ ein sinnvoller Parameter zur
Beurteilung der Stabilität eines Lasersystems ist. Allerdings ist zu beachten, dass das Verhalten am
Rand (nahe $1$) nicht untersucht wurde.


% Moden
Anhand der ermittelten Frequenzen longitudinaler Moden wurden über Gleichung \eqref{eqn:Lichtgeschwindigkeit}
Werte für die Lichtgeschwindigkeit $c$ bestimmt(siehe Tabellen \ref{tab:L1_freq}, \ref{tab:L2_freq}).
Dabei wurde für eine Resonatorlänge von $L = \SI{73.5}{\centi\meter}$ eine mittlere
Lichtgeschwindigkeit von $ c = \SI{3.14(2) e8}{\meter\per\second}$
und für eine Resonatorlänge von $L = \SI{83.5}{\centi\meter}$ eine mittlere
Lichtgeschwindigkeit von $ c = \SI{3.11(2) e8}{\meter\per\second}$ gemessen.
Auffällig ist, dass beide Werte oberhalb der tatsächlichen Lichtgeschwindigkeit
liegen, wobei der Wert für die größere Resonatorlänge($L = \SI{83.5}{\centi\meter}$)
einen kleineren Wert liefert.


% Polarisation



% Wellenlänge
