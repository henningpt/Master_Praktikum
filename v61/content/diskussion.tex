\section{Diskussion}
\label{sec:Diskussion}

% Stabilitätsbedingung
Bei der Untersuchung der Stabilitätsbedingung wurden für $g_{1} \, g_{2}$
Werte zwischen $0.39$ und $0.05$ erreicht.
Wie in Abbildung \ref{fig:oberaffengeilerplot} gut zu erkennen, liegen damit alle Werte
in einem für Stabilität charakteristischen Bereich.
Weiterhin lässt sich grob erkennen, dass die gemessenen Leistungswerte($\propto$Intensität)
im Bereich Bereich um $0.5$ deutlich größer sind als am Rand des stabilen Bereichs nahe $0$.
Somit stimmen die Messungen damit überein, dass $g_{1} \, g_{2}$ ein sinnvoller Parameter zur
Beurteilung der Stabilität eines Lasersystems ist. Allerdings ist zu beachten, dass das Verhalten am
Rand (nahe $1$) nicht untersucht wurde.\\ \\


% Moden
Anhand der ermittelten Frequenzen longitudinaler Moden wurden über Gleichung \eqref{eqn:Lichtgeschwindigkeit}
Werte für die Lichtgeschwindigkeit $c$ bestimmt(siehe Tabellen \ref{tab:freq_L1}, \ref{tab:L2_freq}).
Dabei wurde für eine Resonatorlänge von $L = \SI{73.5}{\centi\meter}$ eine mittlere
Lichtgeschwindigkeit von $ c = \SI{3.14(2) e8}{\meter\per\second}$
und für eine Resonatorlänge von $L = \SI{83.5}{\centi\meter}$ eine mittlere
Lichtgeschwindigkeit von $ c = \SI{3.11(2) e8}{\meter\per\second}$ gemessen.
Auffällig ist, dass beide Werte oberhalb der tatsächlichen Lichtgeschwindigkeit
liegen, wobei der Wert für die größere Resonatorlänge($L = \SI{83.5}{\centi\meter}$)
einen kleineren Wert liefert.\\
Zur Untersuchung der transversalen Lasermoden, wurden die \textbf{$\text{TEM}_{00}$}
und die \textbf{$\text{TEM}_{01}$} Moden untersucht.
Die \textbf{$\text{TEM}_{00}$} Mode konnte(siehe Abbildung \ref{fig:mode00})
beobachtet werden. Der gaußförmige Intensitätsverlauf, dessen Zentrum nahe der
festgelegten Strahlmitte($d = 0$) liegt, ist gut zu erkennen.
Das das Maximum nicht exakt auf der Strahlmitte liegt, kann dadurch begründet werden,
dass die Mitte des Laserstrahls nach Ermessen des Beobachters festgelegt wurde.\\
Zur Untersuchung der \textbf{$\text{TEM}_{01}$} Mode, wurde ein Draht in das
Strahlzentrum gesetzt, sodass die \textbf{$\text{TEM}_{00}$} Mode ausgeblendet wird.
In Abbildung \ref{fig:mode01} ist zu erkennen, dass ein asymetrischer Gaußfit mit zwei Maxima
eine gute Approximation an die Messwerte darstellt. Die Asymmetrie der Maxima, könnte darauf
zurückzuführen sein, dass das Maximum der \textbf{$\text{TEM}_{00}$} Mode nicht exakt
ausgeblendet werden konnte und sich mit einem Maximum der \textbf{$\text{TEM}_{01}$}
Mode überlagert hat.\\ \\


% Polarisation
Bei der Unteruchung der Polarisation des Lasers wurde die Intensitätsverteilung
in Abhängigkeit des Polarisationswinkels untersucht. Dabei konnte, wie erwartet,
beobachtet werden, dass sich eine Periodizität von $\SI{180}{\degree}$
ergibt(siehe Abbildung \ref{fig:polarisation}). Anhand der Messwerte
wurde ein Polarisationswinkel von $\SI{68.3(2)}{\degree}$ bestimmt.\\ \\


% Wellenlänge
Die Wellenlänge der roten Linie im Spektrum des He-Ne-Lasers, wurde mit Hilfe
zweier optischer Gitter ($g_{1} = \SI{80}{\per\centi\meter} \ g_{2} = \SI{100}{\per\centi\meter}$)
gemessen. Für das Gitter mit $80$ Linien pro Zentimeter, ergibt sich eine Wellenlänge
von $\SI{6.53(2) e-7}{\meter}$ und für das Gitter mit $100$ Linien pro Zentimeter
ergibt sich $\SI{6.58(1) e-7}{\meter}$. Es ist zu erkennen, dass sich für das Gitter mit
einer höheren Linienzahl ein geringerer Fehler auf den Mittelwert liefert.
Der Literaturwert\cite{sample} für die Wellenlänge der untersuchten Linie ist
\begin{align*}
  \lambda_{\text{lit.}} = \SI{6.328 e-7}{\meter} \, ,
\end{align*}
sodass sich für die relativen Abweichungen $f_{1}$ und $f_{2}$
\begin{align*}
  f_{1} &= \SI{3.2(3)}{\percent}\\
  f_{2} &= \SI{4.0(2)}{\percent}
\end{align*}
ergeben.
