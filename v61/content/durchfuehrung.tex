\section{Aufbau und Durchführung}
\label{sec:Durchführung}
\paragraph{Justage}
In der Abbildung \ref{fig:aufbau} ist
der Versuchsaufbau dargestellt.
\begin{figure}
  \centering
  \includegraphics{Aufbau.PNG}
  \caption{Versuchsaufbau.}
  \label{fig:aufbau}
\end{figure}
Bevor jedoch mit den eigentlichen Messungen
begonnen werden
kann, muss eine Justage der Bauteile auf der optischen Bank vorgenommen werden.
Zunachst wird der Justagelaser auf die optische Achse
ausgerichtet
mit Hilfe eines Schirm mit Fadenkreuz und Beugungsblende
ausgerichtet.
Dann wird der Austritts Resonatorspiegel
auf die optischen Bank plaziert
und mit den Justierschrauben so ausgereichtet,
dass die erste?? Reflektion wieder auf die
Mitte des Fadenkreuz tifft.
Danach wird der
zweite Resonatorspiegel zwischen Justagelaser
und erstem Spiegel plaziet.
Diesmal wird die zweite?? Reflektion
auf das Fadenkreuz ausgereichtet.
Jetzt kann die Laserröhre mit dem Helium-Neon-Gasgemisch und
den beiden Brewster-Fenster am Ausgang
zwischen beiden Resonatorspiegeln
plaziert werden.
Dann wird der Justierlaser abgeschaltet
und ein Strom an die
Elektroden in der Laserröhre
angelegt, sodass mittels Entladung eine Inversion
stattfindet. Bei Korrekter
Justage setzt die Lasertätigkeit direkt ein
und es bildet sich ein Laserstrahl
zwischen den Resonatorspiegeln aus.
Falls nicht muss mittels den den Justierschrauben
nach justiert werden.
Der Laserstrahl tritt nun aus dem OC(out coupling) Resonatorspiegel
aus und mit einer Photodiode wird die Intensität gemessen.
Die Intensität wird durch nachjustiern der Schrauben maximiert und
es kann mit den Messungen bekonnen werden.



\paragraph{Durchführung}
\paragraph{ Untersuchung der Polarisation}
Zunacht wird die Polarisation des Helium-Neon-Lasers
untersucht. Dafür wird hinter dem OC Spiegel ein
Polarisationsfilter auf die Optische Bank
plaziert und dessen Winkel varriert und mit Hilfe einer Photodiode
die Intensität des Laserstrahls in Abhängigkeit des Winkels bestimmt.

\paragraph{Bestimmung der Wellenlänge}
Die Wellenlänge des HeNe-Lasers wird mit zwei unterschiedliche
Gittern durch ein Beugungsexperimentes aus dem Schirmabstand
und den Abstand der Beugungsmaxima bestimmt.

\paragraph{Unterschuchung der TEM-Moden}
Zunachst werden die longitudinalen Moden untersucht dafür
wird die Resonatorlänge variert und das Frequenzspektrum des Laser
mit einem Frequenzspektrometer aufgenommen.
Falls bei der Variation der Resonatorlänge die Lasertätigkeit
nachlässt, muss mit den Justierschrauben nach geregelt werden.

Dann wird bei fester Resonatorlänge zwei transversale Moden untersucht.
Dafür wird zunacht die $\text{TEM}_{00}$, untersucht indem hinter dem
Laser eine defokussierende Linse plaziert wird und mit Hilfe einer
Photodiode auf einer Schiebebank die Intensitätsverteilungen
durchgemessen.
Für die $\text{TEM}_{01}$ wird zwischen den beiden Resonatorspiegeln
eine dünner Wolframdraht plaziert. Bei genauer Plazierung kann so auf
einem Schirm die gewünschte Mode beobachtet
und wieder Vermessen werden.


\paragraph{Überprüfung der Stabilitätsbedingung}
Um die Stabilitätsbedingung zu überprüfen
wird die Resonatorlänge varriert und jeweils
die Laserleistung durch nachjustiern maximiert und in Abhängigkeit .
der Länge gemessen.
