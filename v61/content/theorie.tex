\section{Theorie}
\label{sec:Theorie}


\subsection{Absorption und Emission}
\label{subsec:absorption_und_emission}

\begin{figure}
  \centering
  \includegraphics{plot.pdf}
  \caption{Plot. \cite{sample}}
  \label{fig:plot}
\end{figure}


\subsection{Konzeptioneller Aufbau des allgemeinen Lasers, sowie des Helium-Neon-Lasers}
\label{subsec:konzeptioneller_aufbau}

Wie schon in Abschnitt \ref{subsec:absorption_und_emission}
beschrieben, besteht ein Laser konzeptionell aus Lasermedium,
Pumpquelle und Resonator.\\
In Abbildung \ref{fig:laserkonzept} ist gezeigt,
wie ein Laser prinzipiell aufgebaut ist.
In Abschnitt \ref{subsec:absorption_und_emission}
wurde bereits auf die das Zusammenspiel von Lasermedium
und Pumpquelle eingegangen.\\
Aufgabe des Resonators ist es nun dafür zu sorgen,
dass der Laserstrahl nach Möglichkeit große Strecke innerhalb
des Lasermediums durchläuft. Dies ist notwendig,
um dem Laserfeld möglichst viel Energie zugeführt
werden kann. Um dies zu gewährleisten besteht der Resonator
aus einem totalreflektierenden und einem teilreflektierenden Spiegel,
deren Spiegelflächen von der optischen Achse durchtoßen werden.
(Siehe Abbildung \ref{fig:laserkonzept})
Auf der Seite des teilreflektierenden Spiegels wird der Laserstrahl
zur Verwendung ausgekoppelt.

In dem in diesem Protokoll beschriebenen
Versuch wird ein Helium-Neon-Laser verwendet.
Bei diesem ist das Lasermedium gegeben






\subsection{Lasermoden}
\label{subsec:lasermoden}
