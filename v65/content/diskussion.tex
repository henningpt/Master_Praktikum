\section{Diskussion}
\label{sec:Diskussion}
In der Tabelle \ref{tab:vergleiche} sind die
bestimmten Messwerte und Literaturwerte und die
relative Abweichung gezeigt.

\begin{table}
  \caption{Bestimmte Werte für $\rho$ und $\delta = n - 1$
  der Schicht(PS) und dem Substrat(Silizium) im Vergleich mit Literaturwerten.}
  \label{tab:vergleiche}
  \begin{tabular}{l l l l}
      \toprule
       Messgröße & Messwerte & Literaturwerte \cite{sample} & rel. Abweichung \\
       \midrule
       $\rho_\text{PS} \text{ in } \si{\per\cubic\meter}$ & \num{4(2)e26} & \num{3.371e29} & \SI{88(5)}{\percent} \\
       $\rho_\text{Silizium} \text{ in } \si{\per\cubic\meter}$ & \num{1.01(1)e30} & \num{7.097e29} & \SI{43(2)}{\percent} \\
       $\delta_\text{PS} $  & $\num{4(2)e-7}$ & $\num{3.5e-6}$  & $ \SI{88(5)}{\percent} $ \\
       $\delta_\text{Silizium}$ & $ \num{10.8(1)e-6} $ & $\num{7.6e-6}$  & $ \SI{42(2)}{\percent} $ \\
      \bottomrule
  \end{tabular}
\end{table}

Die sich ergebendene relativen Abweichungen zu den Literaturwerten aus der Tabelle \ref{tab:vergleiche}
sind deutlich zu groß, um von einer genauen Messung auszugehen.
Mögliche Ursachen liegen zum einen an der Anzahl der Freiheitsgerade bei dem Fit für den
% Korrektur durch Geometriefaktor <-- Hier zu was schreiben

%  -> Fit Detektorscan

%  -> Fit Rockingscan 0 grad

%   -> Korrektur


% Bestimmung der Parameter

%  -> Schichtdicke

%  -> Rauigkeit  <--- Hier zu was schreiben

%  -> Elektronendichte

% -zuviele Fitparameter um eine Genaue Anpassung zu erhalten
%  je nachdem ob an log gefittet wurde oder nicht andere Ergebnisse
%
% - Rockingscan und Detektorscan liefern gewünschte Ergebnisse
%
% - diffuse_Scan nicht wirklicher untergrund aber nur gerninge auswirkung auf den verlauf so
% wie der Gemometrie faktor
