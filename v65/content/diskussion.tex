
\section{Diskussion}
\label{sec:Diskussion}
In Tabelle \ref{tab:vergleiche} sind die
bestimmten Werte für Elektronendichte $\rho$ und $\delta$ und Literaturwerte und die
relative Abweichung gezeigt.
\begin{table}
  \caption{Bestimmte Werte für $\rho$ und $\delta = n - 1$
  der Schicht(PS) und dem Substrat(Silizium) im Vergleich mit Literaturwerten.}
  \label{tab:vergleiche}
  \begin{tabular}{l l l l}
      \toprule
       Messgröße & Messwerte & Literaturwerte \cite{sample} & rel. Abweichung \\
       \midrule
       $\rho_\text{PS} \text{ in } \si{\per\cubic\meter}$ & $\num{3.13(4)e29} $ & \num{3.371e29} & \SI{7(1)}{\percent} \\
       $\rho_\text{Silizium} \text{ in } \si{\per\cubic\meter}$ & \num{6.89(4)e29} & \num{7.097e29} & \SI{3(1)}{\percent} \\
       $\delta_\text{PS} $  & $\num{3.33(4)e-6}$ & $\num{3.5e-6}$  & $ \SI{5(1)}{\percent} $ \\
       $\delta_\text{Silizium}$ & $ \num{7.33(4)e-6} $ & $\num{7.6e-6}$  & $ \SI{4(1)}{\percent} $ \\
      \bottomrule
  \end{tabular}
\end{table}
Die sich ergebendenen relativen Abweichungen zu den Literaturwerten aus Tabelle \ref{tab:vergleiche}
sind zu groß, um von einer genauen Messung auszugehen.
Mögliche Ursachen liegen zum einen an der Anzahl der Freiheitsgerade
bei dem Fit mit dem modifizierten Parrat Algorithmus. Um bessere Ergebnisse
zu erhalten könnte noch in Erwägung gezogen werden, die verschiedenen Bereiche
im Fit gesondert zu gewichten. So könnten relevantere Bereiche besonders
berücksichtigt werden.

Für die Rauigkeit der Polystyrolschicht wurde ein Wert von $ \SI{17.9(5)e-10}{} $
und für die
Rauigkeit des Wafers ein Wert von $ \SI{4.6(1)e-10}{}$ gemessen. Diese liegen im Bereich
der erwarteten Größenordnung. Sie können jedoch nicht mit Literaturwerten verglichen werden,
da die Rauigkeit vom Herstellungsvorgang abhängt.
Desweiteren fällt in Abbildung \ref{fig:messung} auf,
dass der Fit nicht in allen Bereichen an die Messwerte passt, da die Schwingungen
für höhere $\alpha_i$ nicht mehr in der Fitkurve vorzufinden sind.
Dies liegt zum einen an dem verwendeten Zwei-Schichten-Modell,
da Silizium oxidiert und somit in der Realität zwischen PS und Siliziumwafer eine nicht berücksichtigte Siliziumoxid Schicht liegt.
Zum anderen führen verschiedene Startparameter des Fits mit dem modifizierten Parrat Algorithmus,
zu varrierenden Fit-Parametern im Programm.
Dies ist der hohen Anzahl an Freiheitsgraden geschuldet.
 
% Auffällig ist
% der große Fehler auf die Rauigkeit der Polystyrolschicht, sodass der gemessene Wert
% nicht besonders aussagekräftig ist.

% In Abbildung \ref{fig:messung} wird deutlich, dass wenn für
% Elektronendichten und Brechungsindizes die Literaturwerte eingesetzt werden
% und als freie Parameter die Rauigkeit und der z-Wert übrig bleiben, keine
% gute Anpassung an die Messwerte erreicht wird.
% Somit lässt sich vermuten, dass ein systematischer Fehler bezüglich des
% Messvorgangs vorliegt.
%
% \begin{figure}
%  \centering
%    \includegraphics[width=0.7\textwidth]{build/Programm_mit_lit.pdf}
%    \caption{Normierte Reflektivität $R$ gegen den Einfallswinkel $\alpha_i$ aufgetragen und
%     Fit an die markierten Messwerte. Sowie eine Reflektivitätskurve mit den Literaturwerten der Dispersionen und angepasster Rauigkeit.}
%    \label{fig:messung}
% \end{figure}

Abschließend lässt sich sagen, dass bei der durchgeführten Messung, 
Brechungsindizes und Elektronendichten bis auf wenige Prozent genau bestimmt werden konnten.
Weiterhin konnte durch den Versuch die Oberflächenstruktur im Bezug auf die Rauigkeit untersucht werden.
Bessere Ergebnisse lassen sich durch Berücksichtigung der Siliziumoxidschicht und
aufwändigerem Variieren der Schätzparameter beim fitten gewinnen.

% Anpassung an die Messwerte Parameter liefert die stark von den Literaturwerten
% Abweichung beziehungsweise einen großen Fehler aufweisen.
% Dabei ist zu bemerken, dass die verschiedenen Justageschritte viel Potenzial
% für sytematische Fehler beherbergen.


% Korrektur durch Geometriefaktor <-- Hier zu was schreiben

%  -> Fit Detektorscan

%  -> Fit Rockingscan 0 grad

%   -> Korrektur


% Bestimmung der Parameter

%  -> Schichtdicke

%  -> Rauigkeit  <--- Hier zu was schreiben

%  -> Elektronendichte

% -zuviele Fitparameter um eine Genaue Anpassung zu erhalten
%  je nachdem ob an log gefittet wurde oder nicht andere Ergebnisse
%
% - Rockingscan und Detektorscan liefern gewünschte Ergebnisse
%
% - diffuse_Scan nicht wirklicher untergrund aber nur gerninge auswirkung auf den verlauf so
% wie der Gemometrie faktor
