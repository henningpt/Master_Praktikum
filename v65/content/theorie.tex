\section{Theorie}
\label{sec:Theorie}

\subsection{Zielsetzung}
\label{subsec:zielsetzung}
Ziel des im Folgenden beschriebenen Versuchs ist es,
mit Hilfe des Prinzips der Röntgenreflektrometrie
einen Nonometer-dicken Polystyrolfilm, der auf einen Silizium
Wafer aufgetragen ist, zu untersuchen.
Dazu wird ein D8-Labordiffraktometer zunächst justiert und anschließend zu
Messungen verwendet, aus deren Ergebnissen dann Schichtdicke, Rauigkeit und
Elektronendichte der Probe bestimmt werden.


\subsection{Brechung von Röntgenstrahlung an einer Grenzfläche}
\label{subsec:einschicht}
Um das Prinzip der Röntgenreflektrometrie zu verstehen ist es notwendig,
sich mit der Brechung von Röntgenstrahlung an einer Grenzfläche auseinander
zu setzen.

% Brechungsindex
Bei Röntgenstrahlung handelt es sich um elektromagnetische Wellen, deren
Wellenlängen in einem Bereich von circa $\SI{0.1}{\angstrom}$ bis
$\SI{10}{\angstrom}$ liegen.
Um den Übergang einer solchen Welle von einem Medium mit Brechungsindex $n_{1}$
in ein zweites mit Brechungsindex $n_{2}$ (siehe Abbildung \ref{fig:einschicht})
zu beschreiben, kann das Brechungsgesetz nach Snellius
\begin{align}
  n_{1} \cos\alpha_{\text{i}} = n_{2} \cos\alpha_{\text{t}}
  \label{eqn:snellius}
\end{align}
verwendet werden. Dabei bezeichnet $\alpha_{\text{i}}$ den Einfallswinkel
und $\alpha_{\text{t}}$ den Winkel unter dem die Strahlung gebrochen wird
(vgl. Abbildung \ref{fig:einschicht}). Es wird eine ideale glatte Grenzfläche
angenommen. \\
In Abbildung \ref{fig:einschicht} wird für das Ursprungsmedium das Vakuum gewählt,
sodass gilt $n_{1} = 1$. Für den Brechungsindex $n_{2}$ des anderen Mediums,
wird eine komplexe Zahl
\begin{align}
  n_{2} = \alpha + i \beta \, \, \text{mit} \alpha, \beta \in \mathds{R}
\end{align}
gewählt. Der Realteil $\lpha$ beinhaltet die Informationen bezüglich der eigentlichen
Brechung und der Imaginarteil $\beta$ die über Absorption im Medium. Für Röntgenstrahlung
gilt typischerweise $\alpha \langle 1$, da der reele Brechungsindex das Verhältnis
von Lichtgeschwindigkeit im Medium zu der im Vakuum misst und 

% kritischer Winkel


% fresnel -> transmissions/reflektions amplituden -> Fresnelreflektivitaet




\subsection{Brechung von Röntgenstrahlung in Mehrschichtsystemen}
\label{subsec:mehrschicht}

% Reflektivitaet + Kiessig-Ringe


% rekursionszeug + Gesamtreflektivitaet


\subsection{Rauigkeit}
\label{subsec:rauigkeit}





\cite{sample}
